
\documentclass[%
	a4paper,						
	oneside,							
	11pt,
	toc=listof,%					Abbildungs- und Tabellenverzeichnis erhalten einen Eintrag im Inhaltsverzeichnis ohne Nummerierung
	toc=bibliography,%		Literaturverzeichnis erhaelt einen einen Eintrag im Inhaltsverzeichnis ohne Nummerierung
	sprache=english,
	xcolor=dvipsnames
]{refart}%

\usepackage{styles/abschlussarbeit}

% =============================
%	Hauptdokument mit Textinhalt
% =============================
\begin{document}
% --- Titelblatt
% ===========
% Titelseite
% ===========

% --- Kopf- und Fusszeilen der Titelseite sind leer
\thispagestyle{empty}

	
		\begin{center}
			
						\begin{figure}[htbp]
						\begin{minipage}[b][1cm]{0.6\textwidth}
						\includegraphics[width=6cm]{images/WWU_Logo1_1c}
						\end{minipage}
						\hfill
						\begin{minipage}[b][1cm]{0.3\textwidth}
						\includegraphics[width=5cm]{images/microm.jpg}
						\end{minipage}
						\end{figure}
			
				\par
				\vspace*{14ex}
		\Huge
		%			Titel der Abschlussarbeit
					Area segmentation algorithms - \\
					A geoinformatic approach to \\
					Geomarketing strategies\\
				\par
		\normalsize
		\large
					 
			
				\vspace*{15ex}
					Sabine Schmidt\\
					Master of Geoinformatics\\
					- 409156 - \\
					\vspace*{2ex}
					In cooperation with microm Micromarketing-Systems and Consult GmbH\\
				\vspace*{1ex}	
					20th of December 2015
				\par
			\end{center}
			
			\vspace*{25ex}
			First supervisor:\\
			Dr. Torsten Prinz (WWU)\\
			Second supervisor:\\
			Dipl. Ing. Sven Reifegerste (microm)

\newpage

\thispagestyle{empty}
\section*{Declaration}

I hereby declare that I completed this work without any improper help from a third party and without using any aids other than those cited. All ideas derived directly or indirectly from other sources are identified as such. This declaration also refers to the representation of figures and visual material.




\vspace{3cm}
5th of December 2015 \hfill Sabine Schmidt

\pagenumbering{arabic}
\onehalfspacing
\setcounter{page}{3}

\setcounter{secnumdepth}{0}
\titlecontents{section}[2.2em]{\addvspace{1pc}\bfseries}{\contentslabel{2.2em}}{}{\titlerule*[0.3pc]{}\contentspage}

\cleardoublepage
\phantomsection
\addcontentsline{toc}{section}{Abstract}
\section*{Abstract}

Already since several years companies are doing Geomarketing analyses to support their decisions raising their profit and minimizing risks during their product and location planning. Nowadays Geomarketing strategies getting more and more important because more companies realize that they are a helpful tool. To make them more effective concerning time and cost the analyses are done more and more by using computers. But at the moment just a small amount of tools are available which offer functionality of Geomarketing strategies to their costumers. Additionally often the available software considers not all requirements that are defined by Geomarketing experts during the application of area segmentation. Area segmentation processes are clustering small geographic areas to bigger units which are called territories. Doing so the most common conditions are that the created territories need to be balanced, compact and contiguous.  Motivated by the small amount of available software and that the most important conditions of area segmentation are not kept in mind the goal of this thesis is the development of an algorithm supporting area segmentation while considering balanced, compact and coherent territories. Therefore previous work will be analysed for determining promising algorithms. These algorithms will be implemented in order to compare them and determining the most promising approach. The algorithm with best results will be applied to three Geomarketing strategies: area segmentation and optimization, Greenfield analyses and Whitespot analyses. Finally the results will be evaluated and compared to previous work showing advantages and possible disadvantages if exist. 

\cleardoublepage
\phantomsection
\addcontentsline{toc}{section}{Acknowledgement}
\section*{Acknowledgement}

First of all I thank Microm Micromarketing-Systems and Consult GmbH which offer the possibility to do my master thesis within that interesting field of Geomarketing. They have been supported me during the whole time of implementation very well. Additionally microm provided all data which were necessary to do the implementation. Especially I would like to thank Andreas Strade and Sven Reifegerste. \\
Furthermore I thank my first supervisor Dr. Torsten Prinz who helped in case of any questions or problems.\\
Alexander Butsch from KIT Institute of Operations Research provided some information about compactness measures and answered my question quite well. Therefore I would like to thank you.\\
Without the help of my boyfriend and my family the realisation of this thesis would not have been possible. Throughout the entire period producing the thesis they have been supported me very well. I thank all of them very much with special thanks to my boyfriend.


\tableofcontents
%\listoftodos[Notes]
\newpage
\listoffigures
\addcontentsline{toc}{section}{List of Figures}
\newpage 
\listoftables
\addcontentsline{toc}{section}{List of Tables}

\titlecontents{section}[2.2em]{\addvspace{1pc}\bfseries}{\contentslabel{2.2em}}{}{\titlerule*[0.3pc]{.}\contentspage}
\setcounter{secnumdepth}{4} 

\cleardoublepage
\phantomsection
\section{Introduction}

\subsection{Definition and aim of Geomarketing}
The term of Geomarketing has established more and more in the field of marketing within the last years.  A first approximation to the notion of Geomarketing was done in 1995 by  Frühling and Steingrube \cite{fruehling}. They had explained that Geomarketing is just a genus for several instruments within the field of marketing. This shows that Geomarketing is no methodology but rather a discipline. Although some definitions had occurred in the 90s, the first use of Geomarketing analysis went back to the 50s. Already 1952 the first map showing the distribution of purchasing power in Germany was published. In 1982 several companies were founded, that have offered tools and possibilities for their costumers to practise geographic analysis. As a result these researches got easier more and more. Consequently the comprehensive application of Geomarketing was born \cite{herter}. Within the early 90s approaches and fundamentals of geographic analysis and the governance of marketing and distribution has been described within Geomarketing publications. The central idea of Geomarketing is that marketing composed of price, product, distribution and communication will be complemented by space. Consequently operating numbers can change dependent on the spatial location by spatial phenomena of productions and logistics.   


\begin{figureOwn}{Generic consideration of Geomarketing aspects complemented by space \cite{herter}}{images/geomarketing2.jpg}\end{figureOwn}

Although the central idea of Geomarketing is known in the 90s often incorrect explanations are written down while defining the notion of Geomarketing. In several publications it is readable that Geomarketing is a spatial analysis of the market using a geographic information system. But by particularly consideration it is recognizable that this statement is not correct because a geographic information system is just a tool which supports Geomarketing analyses. Herter and Mühlbauer \cite{herter} made a often quoted definition of Geomarketing. They have defined that Geomarketing analyses examine current as well as potential markets considering spatial structures to make the planning of product sales more effective. Additionally the control of the markets should be more quantifiable. That means that all available information about the market are connected to a spatial reference system to make dependences, potentials and other properties visible. The application of Geomarketing analysis have their origins in the minimization of entrepreneurial risks by making the market more transparent so that an purposeful acting is possible. In the course of this, Geomarketing was established as a sub-discipline of the field of marketing. During the application several benefits can be achieved. By knowing potential costumers and competitors the marketing and distribution of products from a company can be done more dedicated so that efficiency enhancement and cost reduction can be caused. Additionally analysis can achive a lead to competitor companies. Furthermore inquests of the market may be helpful during the planning of new locations to determine a site with a high potential so that the risk of a malinvestment can be minimized. It is recognizable that as higher as the number of costumers is, none the worse the benefit of Geomarketing analyses are. During the surveys of the market several principles are utilized. One of them is the spatial factor of the market. Using spatial data (e.g. density of customers, locations, branch offices etc) imporant dependences can be visualized. The spatial data are almost given using addresses. Additionally a spatial heterogeneity can be recognized during analyses. This means that the market differentiates in space. In conclusion a third principle is generated by the mentioned fact. It describes the spatial segmentation of the market. It is recognizable that the higher the number of customers is, the higher the benefits from Geomarketing analyses are. As a prominent example and general speaking consumer in the western part and the eastern part of Germany differentiate in some aspects. In contrast an identification of consumers with similar affectations within a small range of space is possible. From this it follows the neighbourhood principle which explains that neighboured customers have a similar behaviour considering marketing aspects as product purchase. This fact has two reasons. On the one hand costumers with an analogical lifestyle life in the same space and consequently show common characters in costumer behaviour. On the other hand these people share the same infrastructure which takes influences to their purchasing habits as well. Additionally the distance to the location of a company affects the costumers in their decisions whether going to this location or not. 
Geomarketing is based on three stacks: information of the market, geodata and analyses. 

\begin{figurevarSize}{Basic stacks of Geomarketing}{images/stacks.jpg}{0.5}\end{figurevarSize}

Market information are qualitative facts about costumers, competitors etc. within a regional (economic) zone. The data contain information about their socio-demographic, psychosocial, economic and consumption properties like income, product affinity, gender and household size \cite{tappert}. Geodata are information with a spatial reference like addresses, sales areas, locations and catchment areas. The boundaries of these regions may be administrative borders like federal states or townships as well as street sections, coordinates of houses and individual created areas for example a subdivision of postal code areas. By connecting market information to these geodata, analyses are granting important knowledge about the market which are helpful to support companies in their marketing decisions. These facts show that Geomarketing is an instrument for analysing, planning, checking and controlling the market. In the meantime Geomarketing is grown up to one of the most important approaches within the field of marketing to support companies during the accomplishment of their strategies. Consequently it is getting more essential to have systems providing functions and tools which are making these analyses easier and more efficient. 

\subsection{microm Micromarketing-Systems and Consult GmbH}
Microm Micromarketing-Systems and Consult GmbH is a company in Germany which provides Geomarketing analyses to their costumers. It was founded in 1992 and since 1997 it is a subsidiary of the Creditreform. During the last decades microm grew up to one of the biggest providers of Micromarketing and Geomarketing within Germany. It offers possibilities and tools to do anaylses of Geomarketing data. This approach offers the advantage that the company can use all the knowledge which is provided by the employees of microm to control further steps of the marketing decisions. Besides that procedure microm offers additionally a web tool to their costumers so that they can do the analyses by theirself. The software is called mapChart Manager and is accessible with the help of a web browser like Firefox or Google Chrome. 

\begin{figureOwn}{Screenshot of mapChart Manager}{images/mapchart.png}\end{figureOwn}

The advantage of a web tool like the mapChart Manager is that the users can have access to their data and maps from all over the world. Consequently sharing results and working independently from a computer and location makes the application of Geomarketing analyses easier. The mapChart manager offers functionality like the import of data, geocoding of addresses and do anaylses like catchment areas and driving distance zones. To do all that analyses a lot of data is indispensable  like routing networks or information about the behaviour of potential costumers. All these data are offered by microm so that their costumers can buy the information they need. Doing so microm profits from their affiliation to Creditreform which collects costumer data from different resources among other things. As a subsidiary of Creditreform microm can use the information which Creditreform have been collected.

\subsection{Motivation and Research Question}
During the last years the innovation of computer systems raised up so that today a lot of business processes are done by computers to make them more effectiv concerning time and costs. This progress is considerable in the field of Geomarketing, too. More and more software is implemented helping sales managements and business analysts in their decisions. Although already some software packages which provide analyses for Geomarketing strategies are available, the amount of them is really small. Additionally the most of the algorithms containing restrictions concerning several parameters. Per example it may be important that created territories within an area segmentation process become well balanced. Additionally some other constraints may be considered at the same time. The most important condition during area segmentation processes are well balanced, compact and contiguous territories. But no software exists which considers all three parameters. Always just one constraint is kept in mind. That is why the aim of that master thesis is the implementation of an algorithm which considers the three mentioned conditions. Therefore existing algorithms will be considered in detail to determine whether it is possible to find approaches that are promising for the application to Geomarketing strategies. If approaches can be determined the question needs to be answered which algorithm yields the best results concerning balanced, contiguous and compact territories. Afterwards an application of the chosen algorithm will be done to three Geomarketing strategies: area segmentation and optimization, as well as Greenfield analyses and Whitespot analyses.

\subsection{Methods}
For finding a promising algorithm that can be applied to Geomarketing strategies at first previous work will be considered. Therefore algorithms of the past has been analysed to define approaches which will be implemented. As soon as algorithms are determined they will be implemented in JAVA. The results are displayed with QGIS. Afterwards a comparison of the algorithms will be possible using different defined parameters that concern the predefined conditions from the field of Geomarketing. The result of the comparison will show the most promising algorithm that will be applied to area segmentation and optimization, Greenfield analyses and Whitespot analyses. Therefore the algorithm will be enhanced so that the requirements of the analyses will be satisfied. Afterwards an evaluation of the results is done considering advantages and disadvantages as well as algorithms of related work. 

\subsection{Outline}
Within the last decades several institutes and companies implemented area segmentation algorithms for doing their analyses. Getting an insight into some related works at first three examples will be considered within the following chapter. Some common used termini related to Geomarketing will be introduced in section 3 ''Fundamentals of area segmentation''. This section also includes notions and use cases which explain processes of area segmentation in more detail. Afterwards the owned knowledge will be used to consider algorithms from past in more detail to make a selection of promising algorithms possible. Comparing different types of approaches supports the election of approaches. The comparison of approaches and the selection of promising algorithms will be done in section 4 ''Selecting approaches for implementation''. As soon as the selection is done the implementation of the algorithm will be done in section 5. Thereby parameters and data are considered first. Afterwards each algorithm will be explained in more detail, showing their procedure and presenting the results. Using these information a comparison of the implemented algorithms will be done in section 6. Consequently the most promising algorithm can be determined to apply it to Geomarketing strategies. The application of the algorithm to area segmentation and optimization, Greenfield analyses and Whitespot analyses will be shown within the following section. Based on the implementation an evaluation will be accomplished in section 8. Finally the work will be summarized to discuss the results and to state a perspective of further work.
\section{Related Work}

\subsection{KIT - Institute of Operations Research: discrete optimization and logistic}

\subsection{Regio Graph}
\section{Application of area segmentation}

\subsection{Introduction}

inklusive Basisbegriffe, wie base area,...
centren gegeben oder nicht, hier untersuchung auf gegeben,da später anwendung auf geomarketing
\subsection{Parameters}
mögliche Bedingungen
\subsection{previous researches}
- Overview of solution techniques
- split resolution problem: erkenntis: However we found that the running time is still too high for the solution of large-scale problems with many thousands of basic areas in an interactive environment. We can make the allocation step much faster by assigning every basic area to the nearest center. This means that we drop constraints (2). This AllocMinDist heuristic has a greatly reduced running time. However, as one would expect and as the computational results in
Section 7 show, the balance of the territories obtained is not satisfactory. Therefore in the next section we present a new heuristic based on geometric ideas. It
has the desirable property of being very fast (comparable to location-allocation with Alloc-
MinDist) and producing territories that are balanced comparable to location-allocation with
TRANSP and split resolution with AssignMAX. deswegen versucht andere algorithmen zu finden, Quelle: bericht71.pdf



\subsection{Use cases}

\subsubsection{Sales Districting}

siehe gebietoptimalaufteilen pdf seite 29

As each territory elects a single member to a parliamentary assembly, the main
planning criteria is to have approximately the same number of voters in each territory, i.e.
territories of similar size, in order to respect the principle of ”one man–one vote”. bericht71.pdf

If unequal territories exist and if it is generaUy known by the salesmen
that work load or territory potential is disproportionate, this can lead to low morale,
poor performance, a high turnover rate, and an inability to assess the productivity of
individual territories or districts. By rea%ning territories to make them more equitable
with respect to work load or sales potential, a more optimal utilization of each individual
salesman can be achieved.
hessandstuart.pdf

\subsubsection{Political Districting}
The legislative
districting problem—the "one man-one vote" problem—is to subdivide a state into a
specific number of compact and contiguous districts of nearly equal population 
Quelle:hess and stuart


There are three essential characteristics of districts:
the districts should have nearly equal populations to
adhere to the one-person, one-vote principle; the districts
should be contiguous; and the districts should be
geographically compact. We
quelle: mehrotra

Within determined time intervals elections are done within a country to vote for persons who wants to be the representatives of a country, federal state etc. Therefore the area have to be divides into sub parts, so called constituencies. Every constituencies nominees one candidate who will be elected directly into the parliament.  A democratic elections is based one the same weight of every voting that is why some restrictions have to be followed during the political distraction. In Germany these conditions are set down into the Federal Electoral Law §3 Art. 1. It determines that the creation of constituencies should be done in this way that the number of constituencies within the federal state should be agree with the part of the population \cite{bund}. This means that every constituency should hold a similar number of voters compared to other constituencies. The number of inhabitants of Germany is used as stipulation to satisfied that condition. During the distraction the boundaries of townships, districts and cities should be preserved as much as possible. Before an election can be carried out the constituencies has do be proofed and adapted if it is necessary because local alteration of the population can be recognized over the time. A commission will do this in front of every voting. The figure below shows the political distraction in 2013 fr the parliamentary elections.


\begin{figurevarSize}{Politicial Districtings in Germany for parliamentary elections in 2013 \cite{bund}}{images/wahlkreise.jpg}{0.7}\end{figurevarSize}


\subsubsection{Greenfieldanalysis}
\subsubsection{Whitespotanalysis}
\section{Selecting approaches for implementation}\label{Selecting}
\subsection{Requirements of approaches considering geomarketing analyses}

For the application of area segmentation approaches to geomarketing analyses some dedicated requirements are need to be kept in mind to get an useful result. Possible planning criteria where already explained in detail in section \label{criterias} hyperref[criterias]{Additonally Planning criterias}. It was mentioned that contiguity, balance and compactness may playing a huge role during the distriction. considering geomarketing aspects all three planning criteria are need to be satisfied. Taking the example of sales territories owing a certain number of sales man shows that a not balanced alignment yield to an imbalance concerning work load of the employees. This could lead to disaffection. Additionally the territories need to be compact to making the sale process more effective by minimizing travelling time. \\
The area segmentation in this case study will be done to already existing territory centres consequently a determination of new centres in the beginning will be not necessary. \\
All these requirements and conditions needs to be kept in mind during the comparison of model types and the selection of promising algorithm for the implementation.



\subsection{Comparison of model types}
In section \ref{Fundamentals} \hyperref[Fundamentals]{Fundamentals of area segmentation} it was mentioned that there exist three different type of models which where applied in history for distriction processes. The first model type is location-allocation approach and is the one that is mostly used. Within the location-allocation process in the location phase territory centres are chosen. In the second step the basic areas are assigned to these centres. Both steps are done iteratively until a satisfying result of the area segmentation is achieved. For selecting approaches for the implementation within that master thesis that method will be analysed in more detail to determine whether it will be used for that thesis too. The first model of the location-allocation approach was developed by \citeauthor{hess} \cite{hess} in 1965 to solve a political distriction problem. Due to application of the model to sales districting hess model was enhanced by \citeauthor{hessstuart} \cite{hessstuart} and has established known as GEOLINE model. After the implementation they admit that their GEOLINE approach ''does not provide optimal sales territories'' \cite{hessstuart}. Additionally the gained solution my be not well balanced and coherent.
Consequently the practical use of this model is fairly limited. That is why \citeauthor{fleischmann} \cite{fleischmann} developed a modified solution of the GEOLINE model, but their approach demonstrates the same disadvantages like the one of \citeauthor{hessstuart}. The difficulty of the solution of the capacitative transportation problem is the assignment of portions of basic areas to more than one territory centre to satisfy the balancing constraint. Consequently these so called split areas require a more detailed consideration. \citeauthor{hessstuart} \cite{hessstuart} tried to achieve a well balanced solution using a so called AssignMAX approach. Within the alignment they had district the split areas to the territory centres which ''own'' the largest share of the split area \cite{hessstuart, kalcsics}. But \citeauthor{fleischmann} \cite{fleischmann} had proofed that this approach leads to very poor results for their application. Consequently they tried to implement an improved solution of the alignment of split areas but the solution could not resolve all splits automatically thus it was necessary to do some manual post processing \cite{fleischmann, kalcsics}. Due to these problem different improvements were developed to find a good split solution. \citeauthor{schroeder} \cite{schroeder} tried to find a optimal split resolution using tree decomposition. With the help of that approach the best area segmentation in the system of the equations can be determined. Contrary to solving a transportation and split problem \citeauthor{zoltner} \cite{zoltner} implemented an approach using sub gradients. The advantage of that method is the calculation of several possible area segmentation solutions. From this amount of solutions the best one can be chosen. But comparison to the approach implemented by \citeauthor{schroeder} show that the calculation time using the approach of \citeauthor{zoltner} is higher then the one using \citeauthor{schroeder} algorithm \cite{schroeder}. Additionally \citeauthor{zoltner}s approach needs territory centres which are well distributed to achieve a good solution of the calculations. Consequently the algorithms indeed provides coherent areas, but the resulting territories may not be well balanced. Considering all mentioned approaches it can be recognized that no one yields to optimal results considering the area segmentation process. Either the created territories are not coherent or they are not well balanced. Additionally due to the complexity of the split resolution the calculation time is still too high if large scale problems need to be answered \cite{kalcsics}. Furthermore the location-allocation approach is not owing to the linear terms of the equations that need to be answered. Consequently the application of measures of compactness will be constrained considerably. Considering all these problems location-allocation approaches seem to be inapplicable for the application to geomarketing analyses. Consequently such approaches will not be considered in that master thesis. \\
The second type of model is called set-partitioning approach and was implemented \citeauthor{mehrotra}\cite{mehrotra} per example. During the set-partitioning method subdivision of all geographical units will be created. Accordingly these ones will be used to get a balanced result by a partition. Compared to location-allocation methods a major advantage of set-partitioning is the higher flexibility concerning a satisfying result of the area segmentation. In contrary to only limited use of criterion in the location-allocation methods in this approach any criterion can be applied on the generation of candidate districts \cite{kalcsics}. Nevertheless at the same time this advantage is a disadvantage too because the huge flexibility causes a raising combinatoric complexity. That is why the set-partitioning approach can be only used for smaller problems. It have not been used with more than 100 basic areas \cite{kalcsics}. Compared to the location-allocation approaches this method is more ineffective, cumbersome and computationally unattractive \cite{zoltner}. Considering that statement it is obvious that the set-partitioning approach can not be applied to geomarketing analysis to achieve satisfying results. Additionally geomarketing analysis are mostly done to huge area segmentation problems containing a lot of basic areas. Consequently the set-partitioning method can be seen as unusable for that application. \\
The third type of models are heuristic approaches. These ones do not need any solver for linear problems like set-partitioning and location-allocation approaches use. Instead just some mathematical programming is necessary. The main advantage of heuristic methods is the huge flexibility concerning the integration and observance of one or more criterion. At the same time a forecast of the quality of the created territories may be difficulty previously. Normally the quality will be measured afterwards comparing different solutions. Consequently just a relative rating of the quality is possible \cite{schroeder}. Nevertheless compared to the disadvantages of location-allocation and set-partitioning methods heuristic approaches seems to be the most promising ones to apply for geomarketing analysis. By comparing different heuristics it will be tried to find an algorithm with a high quality of the alignment results. To making the algorithm more dependable different heuristics will be combined. Additionally some completely new heuristic approaches will be implemented. In the following table an overview of the advantages and disadvantages of all three types can be found.

\newpage

\setvalue{tableTitle=Comparison of different types of models}

\begin{table}[H]
	\begin{tabular}{|p{2.4cm}|>{\RaggedRight}p{3.2cm}|>{\RaggedRight}p{3.2cm}|>{\RaggedRight}p{3.2cm}|}
		\hline
		 & \centering{location-allocation} & \centering{set-partitioning} & \centering{heuristics} \tabularnewline
		\hline
		\nohyphens{short explanation} & first chose centres, Second assign basic areas to the centres & using subdivisions and partition & mathematical solution of problems \tabularnewline
		\hline
		
		
		advantages & 
		\begin{minipage}[t]{\linewidth}
			\begin{itemize}[nolistsep, noitemsep,after=\strut,leftmargin=10pt,
				before*={\mbox{}\vspace{-\baselineskip}}]
				\item chooses best solution of several calculations
			\end{itemize}
		\end{minipage}
		
		& \begin{minipage}[t]{\linewidth}
			\begin{itemize}[nolistsep, noitemsep,after=\strut,leftmargin=10pt,
				before*={\mbox{}\vspace{-\baselineskip}}]
				\item high flexibility
			\end{itemize}
		\end{minipage} 

		& \begin{minipage}[t]{\linewidth}
			\begin{itemize}[nolistsep, noitemsep,after=\strut,leftmargin=10pt,
				before*={\mbox{}\vspace{-\baselineskip}}]
				\item huge flexibility
				\item do not need a solver
				\item easy to implement
				\item usable for one or several criterion
			\end{itemize}
		\end{minipage} 
\tabularnewline
			\hline
		disadvantages &
		
		\begin{minipage}[t]{\linewidth}
			\begin{itemize}[nolistsep, noitemsep,after=\strut,leftmargin=10pt,
				before*={\mbox{}\vspace{-\baselineskip}}]
				\item territories may not well balanced or not coherent
				\item calculation time still too high caused by split areas
				\item just one criterion applicable
			\end{itemize}
		\end{minipage} 
		
		& \begin{minipage}[t]{\linewidth}
			\begin{itemize}[nolistsep, noitemsep,after=\strut,leftmargin=10pt,
				before*={\mbox{}\vspace{-\baselineskip}}]
				\item useable just for small problems
			\end{itemize}
		\end{minipage} 
		
		& \begin{minipage}[t]{\linewidth}
			\begin{itemize}[nolistsep, noitemsep,after=\strut,leftmargin=10pt,
				before*={\mbox{}\vspace{-\baselineskip}}]
				\item difficult to determine a forecast about the quality of the result previously
			\end{itemize}
		\end{minipage} 
 \tabularnewline
				\hline
	\end{tabular}
\end{table}
 
\subsection{Heuristic approaches}
During the last decades several heuristic algorithms were implemented to solve area segmentation problems. The common one will be considered in that section to chose the approaches that will be taken within this master thesis. \\
\citeauthor{mehrotra} \cite{mehrotra} per example developed an algorithm called \textbf{Eat-up}. During the Eat-up approach one territory after the other is extended at its boundary by adding yet unassigned basic areas to the territory successively. This will be done until the territory satisfying the criteria that need to be balanced. The algorithm was implemented for political distriction with the goal ''to develop a districting method that provides population equality and contiguous and compact districts while retaining jurisdictional boundaries of counties or other political subunits insofar as possible'' \cite{mehrotra}. Within their case study of political distriction in South Carolina they conclude that their implemented algorithm is ''an effective way of generating high quality districting plans'' \cite{mehrotra}. Consequently the Eat-up approach may be one of the potential methods used in that master thesis.\\
\citeauthor{deckro} \cite{deckro} implemented an algorithm called \textbf{Clustering}. That approach treats each basic area initially as a single district. After creating a ranking of neighboured basic areas pairs of districts are merged together iteratively so that a new bigger territory will be created. During the creation of the districts a particular criterion is considered and needs to be satisfied within a range of acceptable variation \cite{deckro}. The districts are merged together until the number of prescribed territories is reached. It was not possible to find further examinations of the usability of that approach. Although it seems to be promising it is not applicable for the area segmentation processes that needs to be done for this master thesis because the distinction will be done using existing territories centres. But the clustering algorithm do not use any centres. \\
\textbf{The Multi-kernel growth} approach was used per example by \citeauthor{bodin}\cite{bodin} in 1977. This method is made up of two steps. In the first step a certain number of basic areas are determined as centres of the territories that should be created. After this step to each centre neighbouring areas are successively added. The neighboured areas are added in order of decreasing distance to the centre. The alignment of basic areas is done until the desired territory size is reached. The second step of this approach is similar to the \textbf{AllocMinDist} method which was implemented by \citeauthor{kalcsics2} \cite{kalcsics2} where the basic areas are allocated to the closest territory centre like it is done in the Multi-kernel growth approach. They conclude that the AllocMinDist algorithm leads to ''disjoint, compact and often connected, however, usually
not well balanced territories as the balance criterion is completely neglected when deciding about the allocation'' \cite{kalcsics2}. Nevertheless they have mentioned that ''the attractiveness of this method [...] lies in its simplicity and computational speed \cite{kalcsics2}. Consequently these results are also adaptable to the second step of the multi-kernel algorithm. The first step can be ignored because in the application of that thesis territory centres are already given. Owing to the mentioned advantages of the allocMinDist algorithm this one may be useful during combining different heuristic approaches. That is why it will be tested during the implementation. Additionally it will be proofed whether the same results of the allocMinDist algorithm are achieved like in the application of \citeauthor{kalcsics2}. \\
Another developed approach for optimization is the so called \textbf{local search}. Within that method the basic areas of neighbouring territories are shifted to minimizing a weighted additive function of different planning criteria \cite{kalcsics}. Consequently an area segmentation needs to be done at first so that the local search can be applied to the solution. Under this aspect local search may be a promising method for combining different heuristics. Per example at first the allocMinDist algorithm may be applied accordingly the local search approach. \\
Another approach that needs to be mentioned is the algorithm which is implemented within the Lizard Library of KIT (see section \ref{KIT} \hyperref[KIT]{KIT - Institute of Operations Research: discrete optimization and logistic} for more information). The \textbf{Recursive Partitioning Algorithm} divides a problem into smaller sub problems until each sub problem satisfies the considered criteria. A similar approach was implemented by \citeauthor{forrest} \cite{forrest}. Problems of the Recursive Partitioning Algorithm were already mentioned in section \label{KIT} so that this approach will not be used within this thesis. Although the created territories seem to be well balanced, they not need to be coherent after the calculation. Additionally the algorithm is based on no territory centres, but the geomarketing application which will be considered first provide existing centres. \\
Considering all mentioned approaches it is recognizable that just some methods seem to be promising while other do not. In conclusion three approaches will be used during the comparison. These approaches are Eat-up,
AllocMinDist / Multi-kernel growth and local search. Additionally some further algorithms are developed. Related to the AllocMinDist approach the same will be done considering only the given criteria that should be balanced. This approach is called AllocCrit in the following. In addition to this several approaches are implemented combining different methods. The first one is appropriated to the AllocMinDist combining with the consideration of the activity measure of the basic areas. Therefore to every territory centre the basic areas are assigned iteratively considering the activity measure of the centre. That means that the centre with the smallest activity measure gets the nearest basic area that is not assigned yet. These steps will be done for each territory centre until all basic areas are allocated. This approach will be called SmallestCritGetsNearest. An improvement of that approach is called SmallestCritGetsTrueNearest which considers neighbouring relationships in more detail. The procedure of the algorithm is similar to SmallestCritGetsNearest but the nearest neighboured basic areas is just allocated if it is really the nearest one to this territory centre. If it is not the case another basic area will be taken. \\ Researches during the last decades show that algorithms that raise the size of the territories from centre may yield to bad results concerning the coherence of the territories. That is why algorithm were implemented where the starting point of alignment lies at the boundary of the observation area. Consequently such a heuristic is implemented within that master thesis too. But this one is combined with the SmallestCritGetsNearest algorithm to yield to better results. Because of the combination of two heuristics the algorithm contains two steps. In the first one all outer territories which lie on the border are allocated to territory centres if they are nearest to one centre. To consider also territories which may be allocated to different territory centres, similar to split areas, a coefficient is used, which should be satisfied for allocation. If the assignment of outer territory centres is done the second step will be initialized which is the SmallestCritGetsNearest approach. The whole algorithm will be called OutsideSmallestCritGetsNearest. The next chosen algorithm is a combination of the Eat-up and the AllocMinDist approaches. Within the Eat-up algorithm each territory centre gets basic areas until the termination criterion build up from the planning criteria is reached. This allocation determines no relationship which basic area will be chosen for the assignment. Within the improved algorithm just the basic areas are allocated which are nearest to the territory centre. The algorithm will be called EatUpMinDist. Furthermore to the mentioned algorithms a combination of AllocMinDist and local search will be implemented too. The algorithm contains two steps similar to the OutsideSmallestCritGetsNearest approach. At first the AllocMinDist processes will be applied. Afterwards the assigned basic areas are rearranged again using local search to create a balanced resolution. Consequently this algorithm is called AllocMinDistLocalSearch\\
All algorithm are implemented with the aim of finding an usable calculation procedure for area segmentation processes for the implementation to geomarketing analyses. All approaches will be compared to find the most promising one. For summarizing the used algorithm they will be shown in the following table again.

\newpage

\setvalue{tableTitle=Overview about selected heuristic approaches for implementation}

\begin{table}[H]
	\begin{tabular}{|p{5.5cm}|>{\RaggedRight}p{7.5cm}|}
		\hline
		& \centering{short explanation} \tabularnewline
		\hline
		AllocCrit & Assigns basic areas to each territory centres dependent on the activity measure of the centre achieving by the sum of the activity measures of the assign basic areas.
		\tabularnewline
		\hline
		AllocMinDist & Assigns the basic areas to the territory centre which is closest.
		\tabularnewline
		\hline
		Eat-up & Extends one territory centre after the other at its boundary by adding yet unassigned basic areas to the territory successively.
		\tabularnewline
		\hline
		SmallestCritGetsNearest & Combination of AllocCrit and AllocMinDist: Assigns the closest yet not assigned basic areas to each territory centres dependent on the activity measure of the centre.
		\tabularnewline
		\hline
		SmallestCritGetsNearest & Improvement of SmallestCritGetsNearest: Process is similar to SmallestCritGetsNearest but the allocation of the closest basic area will be done only if it is the closest territory centre at all.
		\tabularnewline
		\hline
		OutsideSmallestCritGetsNearest & Assigns first basic areas at the boundary of the area under investigation. Accordingly SmallestCritGetsNearest is applied.
		\tabularnewline
		\hline
		EatupMinDist & Combination of Eat-up and AllocMinDist: During the Eat-up approach just basic areas are allocated which are close to the considered territory centre.
		\tabularnewline
		\hline
		AllocMinDistLocalSearch & Combination of AllocMinDist and local search: At first the AllocMinDist will be applied. Afterwards a rearrangement is done using local search to achieve a well balanced solution.
		\tabularnewline
		\hline
	\end{tabular}
\end{table}

\newpage

\section{Implementation of area segmentation approaches}\label{Implementation}
The comparison of different optimization models in the section before shows that heuristic approaches seems to be the most promising methods for the application of that master thesis. That is why 8 different approaches were be chosen which will be implemented. The results of all algorithm will be compared to find the most usable algorithm of that amount.  \\
Section \ref{notions} \hyperref[notions]{Notions and criterias} shows the fundamentals and necessary notions doing an area segmentation.


einleitung: welche daten, plz gebiete (bild welche eigenschafte, wie repräsentiert), hinterlegt mit einem attribut, nicht mehr attribute, warum wahl des testdatensatzes...




\subsection{AllocCrit}

\subsection{AllocMinDist}

pdf gebietoptimalaufteilen, S 150
Ausgehend von den Punkten lassen sich nun Entfernungen div zwischen Zentren
und KGE berechnen. Hierfur gibt es verschiedene Vorschlage in der Literatur. Wir
besprechen kurz die wichtigsten.
Euklidische Distanzen. Die Entfernung zwischen Zentrum i 2 I und KGE v 2 V
ist
$div =(Osti  ostv)2 + (Nordi  nordv)2_1=2$
:
Dies entspricht der Luftlinienentfernung und stellt eine plausible Art der Entfernungsmessung
dar (Cloonan [25]).
Quadrierte euklidische Distanzen. Hierbei ist die Formel zur Berechnung der
Distanz
$div = (Osti  ostv)2 + (Nordi  nordv)2$
Diese Art der Entfernungsmessung erscheint zunachst merkwurdig, da sie kaum als
tatsachliche Entfernung interpretiert werden kann. Dennoch wird sie von vielen Autoren
verwendet (Fleischmann und Paraschis [35], George et al. [42], Hess et al. [50],
Hess und Samuels [51], Hojati [52], Marlin [68]). Nach Einschatzung des Verfassers
gibt es dafur zwei Grunde.
1. In locationallocationModellen ist der location Schritt besonders einfach umzusetzen,
	wenn mit quadriert euklidischen Distanzen gearbeitet wird (vgl. 9.1.1.1).
	Diese Begrundung wird in der Literatur haug zugunsten dieser Art der Distanzmessung
	vorgebracht.
	2. Ebenso wichtig erscheint aber eine Beobachtung, die in 7.3.2 herausgearbeitet
	wird, und die auch im allocation Schritt fur die Verwendung von quadriert
	euklidischen Distanzen spricht. In diesem Fall namlich wird die Ebene in Gebiete
	aufgeteilt, die konvexen Polygonen entsprechen und damit eine Tendenz
	zur Bildung von Bezirken, die der intuitiven Vorstellung von kompakter Form
	entsprechen, geschaen.

dist: He concludes that the success of squared
Euclidean distances depends on the ability to redefine territory centers and is not appropriate
for the case of fixed centers

In most cases, the problem of allocating basic areas to territory centers is formulated as a
capacitated assignment problem, see e.g. Hess et al. [HWS+65] and also Section 5. While
the balancing requirement is generally included as a side constraint, compact and contiguous
territories are tried to be obtained by minimizing the sum of weighted distances between basic
areas and territory centers. For political districting problems, authors tend to use squared
Euclidean distances (e.g. Hess et al. [HWS+65], Hojati [Hoj96]), whereas for sales territory
design problems, largely straight line (Cloonan [Clo72], Marlin [Mar81]) or network distances
(Segal and Weinberger [SW77], Zoltners and Sinha [ZS83]) are employed


\subsection{Eat-up}

\subsection{SmallestCritGetsNearest}

\subsection{SmallestCritGetsTrueNearest}

\subsection{OutsideSmallestCritGetsNearest}

\subsection{EatUpMinDist}

\subsection{AllocMinDistLocalSearch}




\section{Comparison of implemented approaches}\label{comparisonapproaches}
In order to compare the results of all introduced algorithms they have been implemented. The results and a summary of each approach is available in section \ref{Implementation} ''\hyperref[Implementation]{Implementation of area segmentation approaches}''. These results are now used to define parameters that measure the quality of the implemented algorithms. Therefore different terms are defined.
One term is called $\delta$. This one defines the average difference to the balance value that should be reached. The balance value, denoted as $B$ is calculated by the quotient of the sum of activity measures of all basic areas and the number of territories created during the calculation. In the case study the balance value is an activity measure of 34662. Consequently each territory needs to own such a number of basic areas so that this value is approximately reached. For defining $\delta$ the differences of the activity measures of the territories to the balance value are calculated. Afterwards the differences are added and divided by the number of territories. It can be formulated as

\[ \mathit{\delta  = \frac{\sum\nolimits  _{T_{i} \in V} w(T_{i})-B}{N}}\]

Additionally to $\delta$ a second parameter $ \theta $ will be defined. $\theta$ describes the difference to the desirabled number of territories and consequently evaluates the quality of the contiguity of the algorithms. Therefore for each territory the number of sub territories will be calculated. To achieve a satisfying coherency the number of sub territories needs to be one. But the algorithms show that this is not always the case. Consequently each territory can be evaluate concerning the sub territories. These territories can be summed up to determine a conclusion about the coherency. Caused by the fact that the desirabled number of territories needs to be the number of how much territories need to be created (in the case study these are 10 territories), the number of territories can be divided by the number of calculated sub territories to get a quality measure of compactness. Let $n_{T}$ denotes the number of sub territories of each territory then it can be formulated as

\[ \mathit{\theta  = \frac{N}{\sum\nolimits  _{T_{i} \in V} n_{T}}}\]

Besides these two parameters additionally another notion, denoted as $\gamma$, will be used showing the average compactness measure for each algorithm. Therefore the compactness measure of each territory will be summed up. Afterwards they are divided by the number of territories that should be created. Let $c_{T}$ denoting the compactness of the territories. Thus the calculation of $\gamma$ can be formulated as

\[ \mathit{\gamma  = \frac{\sum\nolimits  _{T_{i} \in V} c_{T}}{N}}\]

All three parameters will be used for evaluating and comparing the algorithms. For 
differentiating the distinct sources of basic areas that were used, the parameters getting a footnote,
 \begin{itemize}
 	\item 5 denotes that the calculation was done to zip-code5 areas.
 	\item 8  denotes that the calculation was done to zip-code8 areas.
 \end{itemize}
 
 The results of all algorithms are shown in Table 13. Besides the determined parameters additionally the visualization of the results as well as the calculation time will be shown. \newline
 \\ After summarizing the results the comparison can be done in more detail. Therefore a ranking of the qualities of the different parameters will be done. In section \ref{notions} ''\hyperref[notions]{Notions and criteria}'' the conditions of a satisfying area segmentation process have been defined. It was determined that balance, contiguity and compactness are the three main conditions which should be considered. Consequently the quality of the algorithms can be defined by ranking the results concerning balance, contiguity and compactness. Therefore points between 1 and 10 will be given dependent on the quality of the results. The higher the value, the higher the quality. Afterwards the earned points will be summed up, thus the sums of all algorithms can be compared easily. Defining the ranges for every point the biggest achieved value of a criteria will be taken and divided by 10. Consequently 10 intervals are created. Dependent on the interval a value lies in, the points are given to it. The result of the ranking is shown in Table 14. The calculation time will be not ranked for two reasons. First the main focus is set to the three mentioned parameters. Consequently these are the most important and needs to be chosen to evaluate the algorithms. The second reason is caused by the limited comparability of the performance times. Each algorithm owns a different complexity. This one is influenced by the number of accesses to the databases for example. The database accesses are necessary to get data for solving the area segmentation problem and to perform spatial calculations. Taking AllocMinDistLocalSearch it can be recognized that neighbouring relationships are used to rearrange basic areas to get balanced territories. Consequently a lot of more accesses to the database are necessary to use this relationships. It is obviously that an algorithm without using such relationships is faster during the calculations but may yield to worse results. 
 


\begin{landscape}

\begin{longtable}[H]
	{|p{1cm}|>{\RaggedLeft}p{2.3cm}|>{\RaggedLeft}p{2.3cm}|>{\RaggedLeft}p{2.3cm}|>{\RaggedLeft}p{2.3cm}|>{\RaggedLeft}p{2.3cm}|>{\RaggedLeft}p{2.3cm}|>{\RaggedLeft}p{2.3cm}|>{\RaggedLeft}p{2.3cm}|}
		\hline
		& \centering{AllocCrit} & \centering{AllocMinDist} & \centering{Eat-up} & \centering{SmallestCrit GetsNearest} & \centering{SmallestCrit GetsTrue\-Nearest} & \centering{Outside SmallestCrit\-GetsNearest} & \centering{EatUp MinDist} & \centering{AllocMin DistLocalSearch} \tabularnewline
		\hline
		zip-code5&
		\vspace{1mm}
		\centering{\includegraphics[width=0.18\textwidth]{images/AllocCritplz5.jpg}}
		&
		\vspace{1mm}
		\centering{\includegraphics[width=0.18\textwidth]{images/AllocDistPlz5.jpg}}
		&
		\vspace{1mm}
		\centering{\includegraphics[width=0.18\textwidth]{images/eatupplz5.jpg}}
		&
		\vspace{1mm}
		\centering{\includegraphics[width=0.18\textwidth]{images/smallestcritgetsnearestplz8.jpg}}
		&
		\vspace{1mm}
		\centering{\includegraphics[width=0.18\textwidth]{images/smallestcritgetstruenearestplz5.jpg}}
		&
		\vspace{1mm}
		\centering{\includegraphics[width=0.18\textwidth]{images/Outsidesmallestcritgetsnearestplz5.jpg}}
		&
		\vspace{1mm}
		\centering{\includegraphics[width=0.18\textwidth]{images/eatupDistplz5.jpg}}
		&
		\vspace{1mm}
		\centering{\includegraphics[width=0.18\textwidth]{images/AllocMinDistLocalSearchplz5.jpg}}
		\tabularnewline
		\hline
		zip-code8&
		\vspace{1mm}
		\centering{\includegraphics[width=0.18\textwidth]{images/AllocCritplz8.jpg}}
		&
		\vspace{1mm}
		\centering{\includegraphics[width=0.18\textwidth]{images/AllocDistplz8.jpg}}
		& \vspace{1mm}
		\centering{\includegraphics[width=0.18\textwidth]{images/eatupplz8.jpg}}
		&\vspace{1mm}
		\centering{\includegraphics[width=0.18\textwidth]{images/smallestcritgetsnearestplz8.jpg}}
		& \vspace{1mm}
		\centering{\includegraphics[width=0.18\textwidth]{images/smallestcritgetstruenearestplz8.jpg}}
		&\vspace{1mm}
		\centering{\includegraphics[width=0.18\textwidth]{images/Outsidesmallestcritgetsnearestplz8.jpg}}
		&\vspace{1mm}
		\centering{\includegraphics[width=0.18\textwidth]{images/eatupDistplz8.jpg}}
		&\vspace{1mm}
		\centering{\includegraphics[width=0.18\textwidth]{images/AllocMinDistLocalSearchplz8.jpg}}
		\tabularnewline
		\hline
		$\delta_{5}$ & 2984.8 & 11338.8 & 6932.4 & 3506.4 & 1417.8 & 10.828 & 10145.8 & 1282.4
		\tabularnewline
		\hline
		$\delta_{8}$ & 93.2 & 14945.8 & 927.8 & 125 & 142.8 & 8902 & 1036.8 & 1356.2
		\tabularnewline
		\hline
		$\theta_{5}$ & 0.3448 & 1 & 0.4762 & 0.7143 & 0.7692 & 0.8333& 0.8333 & 0.9091
		\tabularnewline
		\hline
		$\theta_{8}$ & 0.0379 & 0.9091 & 0.5 & 0.2439 & 0.2222& 0.4& 0.5556 & 0.2381
		\tabularnewline
		\hline
		$\gamma_{5}$ & 0.13559 & 0.12793 & 0.12794 & 0.25064 & 0.21807 & 0.24808 & 0.17599 & 0.22664
		\tabularnewline
		\hline
		$\gamma_{8}$ & 0.01535 & 0.15426 & 0.15426 & 0.14309 & 0.08398 & 0.21587& 0.22798 & 0.10337
		\tabularnewline
		\hline
		$t_{5}$ & 0.421 s & 2.707 s & 0.501 s &  0.858 s & 0.779 s & 0.668 s & 1.023 s & 3.525 s
		\tabularnewline
		\hline
		$t_{8}$ & 0.506 s & 3.979 s & 0.583 s & 3.401 s & 3.107 s & 2.669 s & 3.965 s  & 129.920 s
		\tabularnewline
		\hline
		\caption{Overview of results of area segmentation using different heuristics.}\\
\end{longtable}
\newpage


\begin{longtable}[H]	{|p{2.1cm}|>{\RaggedLeft}p{2.1cm}|>{\RaggedLeft}p{2.1cm}|>{\RaggedLeft}p{2.1cm}|>{\RaggedLeft}p{2.1cm}|>{\RaggedLeft}p{2.1cm}|>{\RaggedLeft}p{2.1cm}|>{\RaggedLeft}p{2.1cm}|>{\RaggedLeft}p{2.1cm}|}
	\hline
	& \centering{AllocCrit} & \centering{AllocMinDist} & \centering{Eat-up} & \centering{SmallestCrit GetsNearest} & \centering{SmallestCrit GetsTrue\-Nearest} & \centering{Outside SmallestCrit\-GetsNearest} & \centering{EatUp MinDist} & \centering{AllocMin DistLocalSearch} \tabularnewline
	\hline
	$\delta_{5}$ & 8 & 1 & 4 & 7 & 9 & 1 & 2 & 9
	\tabularnewline
	\hline
	$\delta_{8}$ & 10 & 1 & 10 & 10 & 10 & 5 & 10 & 10
	\tabularnewline
	\hline
	$\theta_{5}$ & 4 & 10 & 5& 8 & 8 & 9 & 9 & 10
	\tabularnewline
	\hline
	$\theta_{8}$ & 1 & 10 & 6& 3 & 3 & 5 & 6 & 3
	\tabularnewline
	\hline
	$\gamma_{5}$ & 2 & 2 & 2& 3 & 3 & 3 & 2 & 3
	\tabularnewline
	\hline
	$\gamma_{8}$ & 1 & 2 & 2& 2 & 1 & 3 & 3 & 2
	\tabularnewline
	\hline
	\textbf{sum} & \textbf{26} & \textbf{26} & \textbf{29} & \textbf{33} & \textbf{34} & \textbf{26} & \textbf{32} & \textbf{37}
	\tabularnewline
	\hline
	\caption{Ranking of different heuristics for area segmentation}\\
\end{longtable}
\end{landscape}

The ranking of the algorithms shows abilities and weaknesses of the approaches. By calculating the sum of the given points it can be determined that AllocMinDistLocalSearch is the most promising algorithm of the amount of implemented approaches. Although the algorithm has some weaknesses yet it associates the best results concerning balance, contiguity and compactness. In the following all algorithm will be compared in more detail to explain the solutions. The smallest sum of points owned AllocCrit, AllocMinDist and OutsideSmallestCritGetsNearest. The problem by using AllocCrit is the missing consideration of coherence and compact territories. Just the activity measures are kept in mind consequently a patchwork rug of territories is created, thus the algorithm is not usable. The same problem can be determined using AllocMinDist but contrary to AllocCrit just contiguity and compactness are kept in mind. Hence the balance is completely ignored so that the resulting territories are mostly coherent but not well balanced. Thus improvement or combinations of different heuristics are necessary to make this approach usable. \\OutsideSmallestCritGetsNearest was used as an improvement of SmallestCritGetsNearest but owns a lot of weaknesses. Dependent on the used threshold value defining the basic areas which are allocated first by considering their distances to the territories centre the result of the algorithm will be more or less balanced. If the threshold is very huge no basic areas will be allocated first so that the algorithm is working equally to SmallestCritGetsNearest. If a small threshold value is taken a lot of basic areas will be assigned in first step. This process causes a great difference concerning the sum of activity measure in each territory. Consequently the tried improvement failed. Thus SmallestCritGetsNearest is more applicable than OutsideSmallestCritGetsNearest. That result can be confirmed showing to the ranking. A little bit more successful than the three mentioned algorithm is the Eat-up approach. Nevertheless that algorithm owns some weaknesses, too. The greatest problem is that it will not be satisfied that even enough territories will be created. By aborting the allocation of basic areas to one territory using a threshold value, for each territories too many basic areas are assigned. Consequently it may be happen that no basic areas can be assigned to the lasts territory centres because all ones were already allocated (see case study using zip-code5 areas).Thus the algorithm can be not used within area segmentation processes using fixed centres. The same problem occurs during the application of EatupMinDist. Within that approach an improvement were implemented to get more coherent territories compared to the ones which were achieved using Eat-up. This target is now satisfied. Nevertheless the creation of the predefined number of territories is not given again. Consequently like Eat-up this algorithm is not usable using fixed centres, too. The results where much better using zip-code8 areas but the algorithm should be more robust in order to fit real world use cases for Geomarketing. But for some data no promising result will be achieved, hence the algorithm is classified to be not applicable. Better results are achieved by using SmallestCritGetsNearest and SmallestCritGetsTrueNearest. Both algorithms almost got the same ranking. There exists just some small differences concerning the given points of balance and compactness. Although both algorithm got similar points within the ranking SmallestCritGetsTrueNearest own a lot of more problems than SmallestCritGetsNearest. During the application of SmallestCritGetsTrueNearest the main problem occurs during the allocation of the basic area that is the closest one. Remembering the algorithm the territory centre with smallest sum of activity measure was determined. Afterwards the nearest basic area which is not allocated yet is chosen. Before allocation it is checked whether another territory centre is located nearer to that basic area. If this is the case the territory centre with smallest activity measure gets the nearest basic area which is allocated in another territory. This change affects an invinite loop because always sane basic area are redistributed circular. A workaround was needed to solve that problem. But the chosen algorithm influences the resolution so much that it is not usable for Geomarketing analyses. Until yet no solution is found which is working dependably. Consequently the SmallestCritGetsTrueNearest approach is not recommended for area segmentation processes yet. If a dependable solution for the infinite loops will be found the algorithm become maybe more promising. Especially the consideration of the distances of each basic areas in more detail is a great advantage to the processes being used in SmallestCritGetsNearest. Because the main problem of SmallestCritGetsNearest is that some basic areas are far away from their territory because just there areas were not allocated yet. Consequently a useful improvement is necessary to make the algorithm applicable for Geomarketing strategies. \\The ranking shows that the most promising algorithm is AllocMinDistLocalSearch. Compared to all other implemented algorithm it delivers the best results. The greatest advantages compared to the others are the two steps which are used in AllocMinDistLocalSearch. With the help of these steps it is tried to find a satisfying solution of balanced and compact territories. Other algorithm like AllocCrit and AllocMinDist considered just one of the parameters whereas AllocMinDistLocalSearch consider both. Nevertheless it is recognizable that there exist some weaknesses concerning compactness and contiguity yet. Consequently improvements are necessary to make it fully usable for the desired analyses. 

\subsection{Requirements from the field of Geomarketing}
Within the chapters before several times requirements from the field of Geomarketing were mentioned. Consequently these will be considered in the following with the regard to the most promising algorithm. The approach with best results is AllocMinDistLocalSearch. \\
The aim during the application of area segmentation process is to define territories which contain a balanced value of an activity measure like household numbers, purchasing power or constituents. Consequently one requirement is the balance of the given activity measure. AllocMinDistLocalSearch uses the local search approach to achieve a balanced result. Within the case study the average difference to the desired activity measure value of each territory was 1282.4 in case of zip-code5 areas and 1356.2 in case of zip-code8 areas. These results show that the territories are well balanced. It is important to check whether such well balanced results will be achieved also during the calculation of other examples to get a convincing evaluation. But at the moment the results satisfying the balance constraint.\\
The second requirement from the field of Geomarketing is contiguity. Considering sales distriction for instance, it is obvious that each sales man needs to have a coherent territory for which he is responsible for. In the most of the cases completely coherent areas are required, too. The results of AllocMinDistLocalSearch shows that satisfying this constraint is not given all the time yet. Concerning the calculation using zip-code5 areas the created territories are not contiguous. The same result is identifiable in the case of using zip-code8 areas. This is caused by the missing consideration of coherence during the local search. Consequently an enhancement of the existing algorithm is necessary which includes also the check of contiguity. This one needs to be implemented to get an applicable algorithm for Geomarketing strategies. \\
Third, compactness is another requirement. Taking again the example of sales distriction compact territories need to be created to minimize travelling times and thus work load. A lot of other approaches also require compact territories consequently it will be determined as one necessary constraint. Concerning the results of AllocMinDistLocalSearch the average compactness is better than the one other algorithms had achieved. Nevertheless the values are just 0.22664 (zip-code5) and 0.10337 (zip-code8). For evaluating the compactness the measure of Cox is used. This one determines that a value between 0 and 1 can be achieved. The nearer the calculated compactness value to 1, the better the compactness is. Both values are nearer to 0 than to 1. Consequently the areas own no good compactness. This fact has two reasons. One problem is the incoherence of some territories because this one takes a huge influence to the compactness, too. Another problem is the shape of the investigation areas. Considering the shape and the containing basic areas it will be not possible to create 10 round territories consequently no compactness measure of 1 can be reached. Nevertheless the compactness values are not satisfying at the moment. Consequently an approach needs to be implemented so that the compactness will be considered much more. \\
Another not yet mentioned requirement for area segmentation process is the location of the territory centre within the territory that belongs to the centre. Some of the implemented algorithms do not provide this requirement, see EatupMinDist for example. Also during the application of AllocMinDistLocalSearch such problem may occur. Consequently a constraint need to be implemented which satisfies that the territory centre is located within the territory.

\subsection{Conclusion}
The comparison of the implemented approaches shows that some of the algorithms are not usable but some of them are. A ranking of different parameters for every use case was done to make the results comparable. Therefore three parameters were defined to summarize the calculation results. One parameter shows the average difference to the balanced activity measure value that needs to be satisfied, the second one considers the contiguity and the third one evaluates the compactness. With the help of these parameters points between 1 to 10 can be given to the algorithms so that a sum of points can be calculated. The algorithm with highest sum will be the most promising approach. The comparison shows that AllocCrit and AllocMinDist are not usable. Additionally the calculations show that Eat-Up and EatupMinDist are not applicable to area segmentation processes using fixed centres, too. OutsideSmallestCritgetsNearest and SmallestCritGetsTrueNearest own too much problems to make them promising for Geomarketing strategies. Just SmallCritGetsNearest and AllocMinDistLocalSearch do have potential to be useful algorithm. The results show that AllocMinDistLocalSearch yields to the best area segmentation solution compared to the other algorithms. Consequently this algorithm will be used for the application to Geomarketing strategies. Before it is fully usable at first some improvements need to be done to fulfil the requirements from the field of Geomarketing completely.

\section{Application of AllocMinDistLocalSearch to Geomarketing analyses}
The aim of this thesis is the comparison of different approaches for are segmentation process to define the most promising one for application to geomarketing analyses. Therefore different approaches were developed and implemented so that a comparison was possible. The comparison shows that the approach called AllocMinDistLocalSearch yields the best results concerning the requirements from the field of geomarketing. During the AllocMinDistLocalSearch algorithm first basic areas are allocated to territory centres dependent of their distance to the centres. Afterwards in second step the basic areas are rearranged again to get balanced territories concerning an activity measure. The implementations showed that there are some improvements are necessary to make AllocMinDistLocalSearch applicable for geomarketing analyses. These ones will be considered during the application and will be explained in more detail within the following sections. The algorithm will be applied to three defined kinds of analyses. One type will be the area segmentation itself which is per example applied during political distriction and by the optimization of existing sales districts. Within these analyses already fixed territory centres are available. The second geomarketing analysis is called Greenfieldanalysis. This ones own no given territory centres. Consequently at first some centres need to be defined. Afterwards the area segmentation algorithm can be applied. Whitespotanalyses are a combination of area segmentation and Greenfieldanalyses. Within the Whitespotanalyses already some territory centres exist. Additionally to the existing ones, new centres need to be created. To all these centres (given ones and new ones) the area segmentation of the basic areas will be done. All three kinds of analyses will be explained in more detail within the following sections. 


\subsection{Area segmentation and optimization}\label{areaseg}
The area segmentation process is the main approach an can be found in both other used type of analyses too. Consequently this algorithm needs to be dependable and need to provide satisfying results because it affects the solution of the other ones too. Area segmentation processes are the heart of all applied districtions per example political distriction, sales distriction and the optimization of sale territories. With the help of area segmentation processes a clustering of basic areas will be done into a predefined number of territories. Thereby one or more specific activity measures need to be considered. Within the case study just household numbers are considered during the allocation. During the comparison of different approach it was determined that AllocMinDistLocalSearch yields to the best results. But the algorithm offers some weaknesses yet consequently enhancements are necessary. Therefore the main ideas of AllocMinDistLocalSearch are taken. The process will be amplified by checking some constraints to satisfying the requirements from the field of Geomarketing. The most important constraint is the location of the territory centres within the territories which belong to it. Therefore it will be determined which basic area contains the territory centre. The resulting basic areas will be saved and considered all the time during the allocation. Remembering the structure of AllocMinDistLocalSearch it was recognizable that it consists of two steps. In first steps baisc areas are allocated to the territory centres dependent on their distance. In second step local search will be applied to realize the balance constraint. Within the first step the centre is always located within the territory even if the constraint is not applied. The problem of lying outside exists first during the rearrangement. Consequently before a basic area will be rearranged it will be checked whether it is the saved basic area, which contains the territory centre. If this is the case, the basic area will not be rearranged. Thus another basic area needs to be chosen. Consequently the basic areas containing the centres are preserved all the time for rearranging. \\
Besides the location of the territory centres, it will be postulated that the created territories need to be contiguous. For satisfying the coherence two different steps are need to implemented. The first step needs to be applied to the AllocMinDist approach which is used in first step of AllocMinDistLocalSearch. In most of the cases the created territories are coherent and compact after that step. But this result is not given all the time. Sometimes some basic areas are allocated in such a way that no contiguity within a territory exists. This problem was already mentioned during the analysis of the implementation of AllocMinDist. Consequently a check needs to be done after the first step of AllocMinDistLocalSearch to satisfying the coherence of the territories. Therefore each created territory will be checked whether all basic areas are contiguous. If it is the case, the next territory will be taken. If the check concludes with an incoherency basic areas will be rearranged to another territory. Therefore the part which is located apart will be defined. For doing so all basic areas of the territory are taken. From the amount of areas firstly the area is chosen in which the territory centre is located. Afterwards all basic areas are determined which can be reached from that area using a graph. All basic areas which are not linked to the amount of basic areas that are reachable are the ones which need to be rearranged. The following figure demonstrates the graph on an example.

\begin{figureOwn}{Using a graph for determining incoherence. All basic areas that can be reached by an edge of the graph (black lines linked by nodes) are contigous. Thus one basic area needs to be rearranged. Example territory using zip-code8 areas.}{images/graphwhitepic.jpg}\end{figureOwn}

After the check all territories are contiguous so that a satisfying base exists for doing the rearrangement by local search. Previously all neighbouring territories are determined to define the one with the highest sum of activity measure. This territory has to deliver one basic area to the territory with the smallest sum of activity measure. Doing so maybe a basic area is given that causes an incoherency in one of the two areas per example by diving the territory that gives the area into two half's. Consequently a check is necessary again whether the basic area can be given. Therefore the same approach is used like after step one. By comparing the resulting graphs of both territories it can be determined whether both ones are contiguous anymore after change. If the contiguity is given for both territories the change can be done. For creating the graph knowledge about neighbouring relationship need to be there. Two basic areas are neighboured if they are sharing edges or points. Determining the neighbour relationships can be done with the help of a function in PostGIS which determines all neighbours of a geometry. These information can be saved and used during the creation of the graph. Thus no additional access to the database is necessary. Another possibility is checking the coherency with the help of a PostGIS function. But this approach needs a lot of additionally database access consequently the run time raises much. Thus this approach was abandoned by performance issues. \\
Additionally to contiguity the compactness of the territories is important. For evaluating the quality of compactness the measure of Cox is used. During the evaluation of the algorithms the compactness measure were just applied afterwards for determining the quality of compactness. Instead of this the measurement will be applied during the rearrangement to get more compact territories. Therefore the compactness of the territories need to be initialized after doing the first arrangement by AllocMinDist in first step. With the help of the initial values of compactness the rearrangement can be done in such a way that the compactness will be considered. Caused by the balance constraint it can be happen that basic areas need to be allocated in such a way that the compactness getting worse. Consequently a solution needs to be implemented to weight both constraints. Therefore a function was created which includes both constraint: compactness and balance. The function measures the change of compactness and balance concerning both values before and after rearranging. The resulting values show whether the rearrangement produces a better resolution or not. Consequently all possible basic areas that may be rearranged can be checked so that the one can be chosen that provides the best result. Let $\Delta \phi $ denotes the change of compactness and and balance it can be written as:

\[ \mathit{\Delta \phi = \phi_{new}-\phi_{old} \to min}\]

$\phi_{new}$ and $\phi_{old}$ need to be calculated using the alteration of balance and compactness including weighting values. The resulting value of $\Delta \phi$ may be smaller or greater 0. If the value is smaller 0, the arrangement yields to a better resolution than before the rearrangement. Vice versa a value greater 0 denotes a degeneration. Consequently the value needs to be as minimally as possible. 
Let $w$ denotes the weightings, $\Delta c$ denotes the change of compactness and $\Delta b$ denotes the change of the balance value thus the calculation of $\phi$ can be formulated as

\[ \mathit{\phi = \Delta c * w_{1} + \Delta b * w_{2} \to min}\]

$\Delta c$ will be calculated using the compactness measure of Cox. The resulting value will be normalized by 

\[ \mathit{\Delta c = 1 - c \to min}\]

The same will be done during the calculation of the change of balance value. For calculating the balance value the difference to the given sum of activity measure each territory should contain will be used. That difference is normalized similar to the compactness measure.

\[ \mathit{\Delta b = 1 - b \to min}\]

The normalizing is necessary for the application of the weighting values. The weighting values can be defined by the user and need to be values between 0 and 1. The application of the algorithm using the weighting function shows that the weighting needs to be for the benefit of balance if the threshold is really small. The threshold is used for defining the allowed difference of the sum of activity measure comparing the territories. At the same time case studies show that a weighting for the benefit of balance does not mean automatically that a worse balanced will be achieved. The weighting value just determines which basic areas will be preferred during the rearrangement. \\ 
The algorithm for area segmentation processes was applied to the same test data which were used during the implementation and comparison of different heuristic approaches. The results of different test cases using several weighting values and threshold will be shown in the following figure.

\begin{figureOwn}{Application of area segmentation process to zip-code5 and zip-code8 areas. a) Weighting values: compactness 1, balance 0, threshold 50. b) Weighting values: compactness 0, balance 1, threshold 30. c) Weighting values: compactness 1, balance 0, threshold 50. a) Weighting values: compactnes 1, balance 0, threshold 100}{images/areasegmentation.jpg}\end{figureOwn}

The results show that the resolution achieved by the area segmentation process depends on the chosen values for the weighting and the threshold. By using the compactness measure the territories are more compact. Their shapes were confirmed by some geomarketing specialist which have considered them as compact. During the calculation of the compactness measure the areas and circumferences of the territories are used. These can be calculated with the help of functions provided by PostGIS. Performance test showed that these calculations need a lot of time because they are done several times during the rearrangement. Consequently a solution was implemented which uses no access to the database. Instead of this the areas, circumferences and shared edges of basic areas were saved in the beginning. Thus only one time a database access is necessarily. Afterwards every calculation can be done with the help of the stored information. \\
The rearrangement of basic areas during the local search offer some problems. It can be recognized in some cases that the predefined threshold will be not reached because always the same basic areas are rearranged again and again so that no correction of the sum of activity measure will be done. Several approaches were tried to find a solution for this problem but no optimal one could be found. Within AllocMinDistLocalSearch the territory with smallest sum of activity measure was determined and took a basic area from the territory with biggest sum of activity measure. Test have showed that the problem of endless rearranging occurs more often in such approach than using the approach which is implemented now. Now the sum of all territories are checked whether the threshold value of balance is satisfied. If one territories does not fulfill the threshold value a basic area will be rearranged. If a territory satisfy the value nothing happens and thus the next territory will be checked. As soon as all territories satisfying the threshold value the algorithm ends. Caused by the endless rearrangement a break needs to be implemented to get a solution in the end. Although other approaches were tried too for optimizing the rearrangement process the implemented one yields to the best results. Nevertheless improvements are necessary to get satisfying results all the time. The workload of the implemented algorithm can be found in following figure. 

\begin{figureOwn}{Workflow diagram of implemnted area segmentation approach}{images/areasegmentation_workflow.jpg}\end{figureOwn}

The initialization process consists of several sub process. Besides the geometries also the neighbours need to be initialized. Additionally centroids of each basic area are calculated used for determining the distances between territory centres using orthodroms. Furthermore areas and circumferences are saved to use them during the rearrangement.

\begin{figurevarSize}{Processes of the initialisation}{images/init.jpg}{0.25}\end{figurevarSize}

\subsection{Greenfieldanalysis}

Greenfieldanalyses are a special type of analyses for planning new locations. When a company is founded at the beginning no locations exists. Consequently it may be needed to determine possible places for location so that a meaningful area segmentation considering an activity measure can be done. Besides the planning of new locations for a company also political distriction using no predefined centres is an example of an approach similar to the Greenfieldanalyses. Doing the analysis it is at first necessary to determine possible locations before the area segmentation can be done. Consequently an algorithm is needed which consists of two steps similar to location-allocation approaches. For defining possible places for the location an approach similar to EatUpMinDist will be used. The comparison of the implemented algorithms showed that this approach is not usable using fixed centres. But doing Greenfieldanalysis in the beginning no centres exists consequently the approach may be promising. The EatUpMinDist algorithm needs to be adapted in such a way that a meaningful starting point can be determined. Therefore a basic area lying on the outer border of the investigation area will be determined. Starting there the first territory will be grew up at the boundary of the territory by allocating basic areas until a threshold value is reached. The threshold value is calculated by summing the activity measures of all basic areas and dividing them through the number of territories that will be created. For creating the next territory the next basic area at the boundary will be chosen which is not allocated yet. Doing so the condition needs to be satisfied that the basic area is neighboured to the territory that was created before. As soon as a basic area was found the territory can be created. These will be done until the predefined number of territories is created. If no basic area at the boundary exist anymore another basic area is chosen which is neighboured to the territory created before independently on the location of that area. If no not yet allocated basic area which is neighboured to the territory is available anymore, a basic area will be taken which is not neighboured but yet not allocated. Following these steps all territories can be created. The following figure shows the created territories after the first step using the case study investigation area.

\begin{figurevarSize}{Created intial territories for defining initial territory centres}{images/greenfieldfirst.jpg}{0.8}\end{figurevarSize}


The problem using this approach is given by the threshold value to abort the allocation of basic areas to one territory. It was already mentioned during the explanation of Eat-up and EatUpMinDist. By exceeding the threshold value for abortion it may happen that already all basic areas are allocated before the predefined number of territories is reached. To prohibit this phenomena additionally abortion criteria need to be defined to ensure all territories contain at least one basic area. This can be done by checking the resulting number of not yet allocated basic areas. \\
After the creation of initial territories is done territory centres can be determined by setting the centres into the middle of each territory. Afterwards these centres can be used doing the area segmentation process. The used area segmentation algorithm is the same which is used in section \ref{areaseg} \hyperref[areaseg]{Area segmentation and optimization}. This one will be applied using the predefined territory centres. As soon as the area segmentation is passed final territory centres will be determined. The following figures show the workflow and results of that calculation using different values for weighting and threshold. In both application 10 territories and thus 10 locations are created.


\begin{figurevarSize}{Workflow of algorithm used for Greenfieldanalyses}{images/greenfield_workflow.jpg}{0.3}\end{figurevarSize}

\begin{figureOwn}{Results of Greenfieldanalyses. a) Zip-code5 areas. Weighting values: compactness 1, balance 0, threshold 50. b) Zip-code8 areas. Weighting values: compactness 0.1, balance 0.9, threshold 50.}{images/greenfieldend.jpg}\end{figureOwn}

\subsection{Whitespotanalysis}

1.	Erzeugung der Gebiete der gegebenen Standorte
2.	Erzeugung der Gebiet für neue Standorte --> Erzeugung von Startlocations
3.	Reset Areas; use all locations (new + given) and do area segmentation

Zwei verschiedene Ansätze bei 2.:
-	Entweder Nutzung eines Randpolygons als Start zur Erzeugung des Gebietes
-	Oder Nutzung eines beliebigen noch nicht verteilten Polygons: Problem: Punkte/Standorte liegen evtl zu dicht beieinander und dann auch noch zu viele dicht nebeneinander

Beide Whitespotansätze sind nun implementiert. Es ist erkennbar, dass der Ansatz mit Nutzung von Randpolygonen das bessere Ergebnis liefert: Rechenzeit ist kürzer + Gebiete sind besser geformt, also wird dieser Ansatz genommen.

\section{Evaluation}\label{evaluation}

After all implementations were done an evaluation of the used algorithm was accomplished. Therefore different investigation areas are used applying several parameters to make an analysis of the results possible. The results have been compared with results from other Geomarketing software. \\
First a comparison to a calculation in SIM Tool was done. Therefore microm provided some results doing a Greenfield analysis to zip-code8 areas of Hamburg. The area segmentation process was done using approximately 2500 basic areas. Thereby 200 new locations should be created. Afterwards the basic areas should be allocated to these locations. The calculation using SIM runs 1:30h. The same calculation was done using the Greenfield algorithm which was implemented within this thesis. For making the results comparable the same dataset was used. The aim of the implemented algorithm was to create coherent, compact and well balanced territories within an adequate running time. The results of the Greenfield analysis using the algorithm implemented for this thesis achieved the result within 5 minutes.. Contrary to the territories which were created by SIM, the created territories were all contiguous. Additionally the territories were more balanced than the ones of SIM. Nevertheless the balance can be optimized further. The result of the calculation using the algorithm implemented in this thesis is shown in Figure 47.

\begin{figure}[H]
	\centering
	\includegraphics[width=0.8\textwidth]{images/HH10.jpg}
	\caption[Result of Greenfield analysis doing to zip-code8 areas of Hamburg.]{Result of Greenfield analysis doing to zip-code8 areas of Hamburg. 10 locations and territories are created. The check of unity was not used in that calculation.}
\end{figure}


Additionally a comparison to another Geomarketing software called Map\&Market was done. Thereby whole Germany was used as investigation area, consequently approximately 8000 basic areas were assigned to the locations. The area segmentation was done using a Greenfield analyses creating 10 new locations. Doing this calculation a lot of problems occurred. The biggest problem are the islands in the North Sea and Baltic Sea. The islands are not connected to any other basic areas. Consequently they are not continuous to a territory in sense of the definition used in that thesis because no shared edges of two basic areas exists. That is why the check of contiguous yields to errors. To solve that problem a variable was integrated that shows whether the check of coherence should be done or not. In case of using islands within the amount of basic areas no totally contiguity can be reached thus the variable needs to indicate that no check will be done. Although the check of coherence was disabled during the calculation most of the basic areas are allocated in such a way that they are contiguous. The algorithm was improved thus the possibility creating incoherent territories become smaller. Thus the results form an expected output. The result is illustrated in Figure 48. 

\begin{figure}[H]
	\centering
	\includegraphics[width=0.9\textwidth]{images/dtl.jpg}
	\caption[Result of Greenfield analysis doing to zip-code5 areas of Germany.]{Result of Greenfield analysis doing to zip-code5 areas of Germany. 10 locations and territories are created. The check of unity was not used in that calculation. Parameters: compactness 1, balance 0, threshold 30.}
\end{figure}

Although a lot of improvements concerning the performance were done the calculation time applying the Greenfield algorithm to whole Germany is still too high. This is caused by two facts. One fact are the number of database accesses which are necessary during the calculation. Before the area segmentation starts the polygons including their properties and neighbouring relationships are stored into local variables. Using 8000 basic areas these preparatory works needs up to half an hour. After this the allocation starts which needs running time again. Especially the local search needs a lot of time, thus the calculation time is higher than the one of Map\&Market. Map\&Market needs 50 minutes, the algorithm which was implemented here needs 1:20h. Nevertheless the approach of the thesis owns some advantages compared to the one used in Map\&Market. Thus the compactness is considered and contiguity will be achieved if the data is applicable doing so. However some more improvements are still necessary. Some preparation within the database may yield to faster calculations as well. During the calculation of the two test cases some problems could be identified especially concerning performance. To make the algorithm much faster a lot of improvements were done. First of all the number of accesses to the database were reduced. Therefore a lot of queries were removed. Instead of this just one access to the database in the beginning is used now which stores all necessary data. This one is just one example of enhancement that were done. \\
Additionally to the two comparisons of use cases the algorithms were tested to different investigation areas using several parameters. Thereby some weaknesses could be determined. Especially the implementation of the local search yields to some problems. They are caused by rearranging same basic areas again and again so that no better balance will be achieved. That is why an abortion was implemented after a determined running time or number of rearrangements. This problems leads to two weaknesses. First it may be happen that the predefined balance threshold will be not reached. Consequently the territories are not well balanced. Second the running time is maybe still high although no better balance is achieved. That is why an advancement of that algorithm is still necessary in future work. However the implemented algorithms for area segmentation, Whitespot analyses and Greenfield analyses are good prototypes doing area segmentation. Compared to other applications they yield to convincing results within an adequate running time. Nevertheless some improvements are necessary concerning the balance and the performance. Additionally it is not satisfied that an optimal solution of area segmentation is found, but therefore the running time is still fast. 


\section{Discussion and Perspective}
\subsection{Summary}
\subsection{Limitations}

gemeinsame kante für anchbarschaftsbeziehungen evtl ungeegeignet, da dabei keine geographischen grenzen wie flüße oder gebirge beachtet werden

\subsection{Comparison to related work}

zu algorithmus von kit vergleichen:
vorteil: beste lösung wird ermittelt
nachteil: bei viele daten expoentiell großer baum, nur auf gebietsverteilung ohne standorte, um rechenzeit einzsuchränken nur limitierte anzahl an richtungen der Linien + nur geraden, somit leidet die Kompaktheit, außerdem verschnitt auf PLZ-Gebiete notwendig




\subsection{Perspective}

vergleichen zu algorithmus von gebieteoptimalaufteilen

%\begin{appendix}
%\include{chapters/appendix}
%\end{appendix}


\newpage
\addcontentsline{toc}{section}{References}

\nocite{*}
% legt die zu verwendete BibTeX Stildatei fest; einige Stildateien gehoeren zu Erweiterungspaketen, die ggf. zusaetzlich einzubinden sind
\bibliographystyle{alpha}
% Befehl ist an der Stelle zu verwenden, an der das Literaturverzeichnis gesetzt werden soll
\bibliography{bib/verzeichnis}



\end{document}
