\section{Related Work}

\subsection{KIT - Institute of Operations Research: discrete optimization and logistic}\label{KIT}

The Karlsruhe Institute of Technology (KIT) was founded in 2006 by the amalgamation of the university of Karlsruhe and the research center Karlsruhe. Together both institutes form one of the biggest research and education constitution in the world which concentrates on selected research areas. On the Institute of Operations Research the staff members delve into the field of discrete optimization and logistic which includes districting, too. In progress of studies they have been implemented a library called ''Lizard'' which includes an algorithm for the solution of area segmentation problems. The library can be downloaded for free, so an execution on given example data is possible. In addition to this, it is implemented into a Geographic Information System in the web browser. The algorithm of Lizard is based on the so called ''Recursive Partitioning Algorithm'' which was designed by Kalcsics et al. \cite{kalcsics}. The algorithm is a geometrical approach which divides the problem into sub problems to align territories that are balanced with respect to an activity value. The partition is done by placing a line into the dataset. Consequently a left problem (data left from line) and a right problem (data right from line) are created. Within the next steps additional lines are placed into the sub problems to divide them again into smaller ones. The number of sub problems depends on the number of distracted areas which should be created. The lines may have different directions which will be considered. These are considered all so that different partitions of sub problems exists. The sub problems are described by a binary tree whose root element complies the problem that should be solved. The different partition for the subdivisions are ranked by a heuristic measure for balance and compactness so that the partition with the best measure value can be chosen as result.


\begin{figure}[H]
	\centering
	\includegraphics[width=0.8\textwidth]{images/kit_alg.jpg}
	\caption[Possible partitions within ''Recursive Partitioning Algorithm''.]{a) Possible partitions using vertical lines in ''Recursive Partitioning Algorithm'', b)Two possible partitions with different compactness \cite{kalcsics}}
\end{figure}

\newpage
The steps of the partitions are presentable within Lizard. The download package of the algorithm already includes such example files which can be executed. The examples contains either point coordinates or line segments for visualizing street segments. Both input types are linked to an activity measure which will be used to achieve balanced territories. It is recognizable that no given centres exists. Consequently the area segmentation is not done considering given locations like. Before the algorithm in Lizard can be started some parameters need to be chosen. There exist input options for:
\begin{itemize}
	\item Number of areas: This number defines how many territories will be created.
	\item Number of line directions: This value is necessary to define the allowed positions of each line which will be placed into the problem for dividing into smaller parts.
	\item Balance Tolerance: The value determines the range which needs to be satisfied comparing the activity measure of the territories.
	\item Weight Balance: The value needs to be between 0 and 1 to determine the importance of a balanced activity measure of the territories. In case of 1 balance is most important, in case of 0 the balance is not important. If balance is defined as 0, then compactness is most important.
	\item Compactness Measure: There exist different options how the compactness value is calculated. The name of some options are ConvexHullIntersection, MaximumDistance and WeightedPairwiseDistance. ConvexHullIntersection for example uses the circumference of the convex hulls of all territories during the calculation of the compactness \cite{kit_lizard}. Consequently every mentioned option defines another approach of the calculation of the compactness rate.
	\item Bisecting Partition: With the help of this option the geometric object is selectable which divides the problem. The predefined option are lines like it is defined within the ''Recursive Partitioning Algorithm''. Additionally FlexZone and a combination of both can be chosen.
\end{itemize}

The visualisation of the result of the districting problem can be done step by step or by showing just the final result. Showing the results in partial processes represents the approach of the algorithm dividing the problem into sub problems. The calculation of the tree is not recognizable, but the leaf with the best measuring value and thus the best calculation will be shown. Additionally information about other partition solutions are shown, too. In Figure 5 the algorithm is visualized for a area segmentation problem where street segments should be divided into four balanced territories. The feasible balance tolerance is 0.1, the number of search directions is 8, the weight for the balance is 0.5 and the bisecting partition is done by a line. The results of the activity value for every subregions are shown in Table 1.

\begin{table}[H]
	\centering
\begin{tabular}{|*4{C{3cm}|}}
	\hline
	Blue territory & Red territory & Violette territory & Orange territory \tabularnewline
	\hline
	12224 & 12210 & 12216 & 12208 \tabularnewline
	\hline
\end{tabular}
\caption{Activity measures of resulting areas after doing an area segmentation using ''Lizard''}
\end{table}


\begin{figurevarSize}{Visualization of area segmentation using ''Lizard''}{images/kit_liz.jpg}{0.7}\end{figurevarSize}

Dependent on different settings concerning the available options during the calculations different results can be achieved. Some examples are shown in the following figure. All calculations are done on the same data set which was already used in Figure 5. To emphasize the impact of the input parameters they have been changed. Their results are displayed in the sub figures of Figure 6. In Figure a) the importance of balance was set to 0, instead the compactness of the resulting territories should be most important. In Figure b) the options are chosen the other way around, consequently the compactness is less important and the balance is most important. Figure c) shows the result of the disctricting problem if just one search direction for the lines which were put into the dataset is allowed.

\begin{figurevarSize}{Results of area segmentation by ''Lizard'' using different options}{images/kit_lizard_opt.jpg}{0.9}\end{figurevarSize}

\subsection{Easy Map District Manager}
The Lutum + Tappert DV-Beratung GmbH is one of the biggest companies in Germany which provides tools, data and services to do Geomarketing analyses. The company was founded in 1982. Already 1986 they sold for the first time a software for doing Geomarketing analyses, which is called Easy Map District Manager. Thenceforth the software was emphasized by a lot of functions. Today Lutum + Tappert provide different attendances from the field of Geomarketing. The main field is the sale and application of their software ''Easy Map District Manager''. With the help of this data of costumers, potentials of markets and data of the population can be visualized for example. Thus a costumized analysis is possible to realize marketing strategies, control the market and check some statistics. Additionally to the software the company offers data which may be necessary to acquire new costumers or to do more detailed analyses. Furthermore they offer services to help companies during their researches. \\
The most important tool considering related work is the software Easy Map District Manager. This one is provided in two different versions, which include different options and analysis functionalities. The Easy Map Standard Desktop Edition (EMSDE) is useful to create maps and to do some geographical analyses. However the EMSDE may be necessary to do area segmentation and planning of locations. Both editions are available as a demo version, consequently a deeper insight into the functionality of the software is possible. After the installation of the District Manager some sample data can be used to do some analyses. Therefore zip-code areas of the city of Hamburg and the neighbourhood of it are taken. The tool offers three possibilities to do area segmentation. The first one generates new territories if no distriction was done before. Within the test scenario a given number of territories are created by allocating the zip-code areas to districts. The number of zip-codes areas within every district should be as balanced as possible compared to the others. After the allocation is done locations are set into the middle of the territory to show the position of new company sites. This approach is similar to the Greenfield analyses which will be explained later. The second functionality uses an distribution of areas that already exists. The containing zip-code areas in every superior area will be rearranged if it is necessary to get a balanced area segmentation concerning the number of areas within every district. Existing locations within that dataset are not heeded during the rearrangement. The third option uses just existing locations. With the help of these sites the zip-code areas are allocated in this way that every location is placed within one district. Additionally every district should have the same number of zip-code areas again. These two provided functions are both from the field of the optimization of area segmentation. In reality often additional data are used during the rearrangement for instance the number of households or the purchasing power. These information are used during the area segmentation and are often the parameter which should be balanced. Such a use case was not possible to test because such data are not available for the chosen area. Nevertheless an analysis concerning the number of postcode areas can be done so that an inspection of the results is possible. It is recognizable that the resulting territories are mostly coherent but there is no need to get an absolutely contiguous district. Especially at borders to other regions some postcode areas are located inside of the neighboured region. Consequently it is not connected directly to the superior district which it belongs to. Thus the conclusion can be achieved that no checking of coherence will be done. By that reason the islands of the north sea do not lead to some problems during the allocation. 

\begin{figure}[H]
	\centering
	\includegraphics[width=0.9\textwidth]{images/easymap.jpg}
	\caption[Result of creating 20 areas with Easy Map District Manager concerning a balanced number of zip-codes areas for each territory.]{Result of creating 20 areas with Easy Map District Manager concerning a balanced number of zip-codes areas for each territory. The clipping shows incoherent territories.}
\end{figure}


After doing the area segmentation the locations are placed into the middle of the appropriate territory. In this case no other placement can be chosen. Just if an optimization of company sites should be done the locations can be placed into the weighted centroid of the district, too. Furthermore it is recognizable that no new locations can be added to the area where already a certain number of sites exist. But this will be necessary if a new company site needs to be opened. This fact will be described later on with the help of the term Whitespot analyses. Nevertheless the Easy Map Manager offers a lot of possibilities to do some Geomarketing analyses.  

\newpage
\subsection{SIM Tool}
SIM Tool is a tool for performing analyses of location information and managing existing locations. It was build by cooperation of microm and digital data services GmbH. The tool provides functionality for its users concerning topics that deal with locations of companies. Therefore simulations can be done to consider effects of changes of these sites. Additionally consequences of a movement of a location or a creation of a new one by competitors can be analysed. By specializing topics that deal with locations consequently Greenfield and Whitespot analyses are included as well. Additionally calculations can be done determining the portion of potential at the market. \\
The SIM Tool is compared to the outcome of that thesis in chapter \ref{evaluation} ''\hyperref[evaluation]{Evaluation}''. Therefore the results of a Greenfield analysis in Hamburg will be considered. For analysing the outcome of area segmentation processes using SIM a calculation is done creating 20 new locations in Hamburg. Therefore zip-code areas are used to be clustered into territories. The result of the calculation is shown in the following figure. \\
The figure shows that SIM Tool do not check the contiguity of the territories. That is why some zip-code areas are located aside of the territory like it is recognizable at the clipping. Additionally it may happen that the territories are not well balanced. The analyses of the tool show, that no threshold value is used. Consequently the balance of the territories will not be satisfied. Furthermore no rearrangement of zip-code areas is done in order to get a better balance. Nevertheless the results of SIM convince a lot of Geomarketing experts so that they are often used for Geomarketing analyses.


\begin{figure}[H]
	\centering
	\includegraphics[width=0.9\textwidth]{images/simtool.jpg}
	\caption[Results of a Greenfield analysis in SIM Tool creating 20 new locations in Hamburg.]{Results of a Greenfield analysis in SIM Tool creating 20 new locations in Hamburg. The clipping shows incoherent territories.}
\end{figure}

