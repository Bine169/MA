\section{Introduction}

\subsection{Definition and aim of Geomarketing}
The term of Geomarketing has established more and more in the field of marketing within the last years.  A first approximation to the notion of Geomarketing was done in 1995 by  Frühling and Steingrube \cite{fruehling}. They had explained that Geomarketing is just a genus for several instruments within the field of marketing. This shows that Geomarketing is no methodology but rather a discipline. Although some definitions had occurred in the 90s, the first use of Geomarketing analysis went back to the 50s. Already 1952 the first map showing the distribution of purchasing power in Germany was published. In 1982 several companies were founded, that have offered tools and possibilities for their costumers to practise geographic analysis. As a result these researches got easier more and more. Consequently the comprehensive application of Geomarketing was born \cite{herter}. Within the early 90s approaches and fundamentals of geographic analysis and the governance of marketing and distribution has been described within Geomarketing publications. The central idea of Geomarketing is that marketing composed of price, product, distribution and communication will be complemented by space. Consequently operating numbers can change dependent on the spatial location by spatial phenomena of productions and logistics.   


\begin{figureOwn}{Generic consideration of Geomarketing aspects complemented by space \cite{herter}}{images/geomarketing2.jpg}\end{figureOwn}

Although the central idea of Geomarketing is known in the 90s often incorrect explanations are written down while defining the notion of Geomarketing. In several publications it is readable that Geomarketing is a spatial analysis of the market using a geographic information system. But by particularly consideration it is recognizable that this statement is not correct because a geographic information system is just a tool which supports Geomarketing analyses. Herter and Mühlbauer \cite{herter} made a often quoted definition of Geomarketing. They have defined that Geomarketing analyses examine current as well as potential markets considering spatial structures to make the planning of product sales more effective. Additionally the control of the markets should be more quantifiable. That means that all available information about the market are connected to a spatial reference system to make dependences, potentials and other properties visible. The application of Geomarketing analysis have their origins in the minimization of entrepreneurial risks by making the market more transparent so that an purposeful acting is possible. In the course of this, Geomarketing was established as a sub-discipline of the field of marketing. During the application several benefits can be achieved. By knowing potential costumers and competitors the marketing and distribution of products from a company can be done more dedicated so that efficiency enhancement and cost reduction can be caused. Additionally analysis can achive a lead to competitor companies. Furthermore inquests of the market may be helpful during the planning of new locations to determine a site with a high potential so that the risk of a malinvestment can be minimized. It is recognizable that as higher as the number of costumers is, none the worse the benefit of Geomarketing analyses are. During the surveys of the market several principles are utilized. One of them is the spatial factor of the market. Using spatial data (e.g. density of customers, locations, branch offices etc) imporant dependences can be visualized. The spatial data are almost given using addresses. Additionally a spatial heterogeneity can be recognized during analyses. This means that the market differentiates in space. In conclusion a third principle is generated by the mentioned fact. It describes the spatial segmentation of the market. It is recognizable that the higher the number of customers is, the higher the benefits from Geomarketing analyses are. As a prominent example and general speaking consumer in the western part and the eastern part of Germany differentiate in some aspects. In contrast an identification of consumers with similar affectations within a small range of space is possible. From this it follows the neighbourhood principle which explains that neighboured customers have a similar behaviour considering marketing aspects as product purchase. This fact has two reasons. On the one hand costumers with an analogical lifestyle life in the same space and consequently show common characters in costumer behaviour. On the other hand these people share the same infrastructure which takes influences to their purchasing habits as well. Additionally the distance to the location of a company affects the costumers in their decisions whether going to this location or not. 
Geomarketing is based on three stacks: information of the market, geodata and analyses. 

\begin{figurevarSize}{Basic stacks of Geomarketing}{images/stacks.jpg}{0.5}\end{figurevarSize}

Market information are qualitative facts about costumers, competitors etc. within a regional (economic) zone. The data contain information about their socio-demographic, psychosocial, economic and consumption properties like income, product affinity, gender and household size \cite{tappert}. Geodata are information with a spatial reference like addresses, sales areas, locations and catchment areas. The boundaries of these regions may be administrative borders like federal states or townships as well as street sections, coordinates of houses and individual created areas for example a subdivision of postal code areas. By connecting market information to these geodata, analyses are granting important knowledge about the market which are helpful to support companies in their marketing decisions. These facts show that Geomarketing is an instrument for analysing, planning, checking and controlling the market. In the meantime Geomarketing is grown up to one of the most important approaches within the field of marketing to support companies during the accomplishment of their strategies. Consequently it is getting more essential to have systems providing functions and tools which are making these analyses easier and more efficient. 

\subsection{microm Micromarketing-Systems and Consult GmbH}
Microm Micromarketing-Systems and Consult GmbH is a company in Germany which provides Geomarketing analyses to their costumers. It was founded in 1992 and since 1997 it is a subsidiary of the Creditreform. During the last decades microm grew up to one of the biggest providers of Micromarketing and Geomarketing within Germany. It offers possibilities and tools to do anaylses of Geomarketing data. This approach offers the advantage that the company can use all the knowledge which is provided by the employees of microm to control further steps of the marketing decisions. Besides that procedure microm offers additionally a web tool to their costumers so that they can do the analyses by theirself. The software is called mapChart Manager and is accessible with the help of a web browser like Firefox or Google Chrome. 

\begin{figureOwn}{Screenshot of mapChart Manager}{images/mapchart.png}\end{figureOwn}

The advantage of a web tool like the mapChart Manager is that the users can have access to their data and maps from all over the world. Consequently sharing results and working independently from a computer and location makes the application of Geomarketing analyses easier. The mapChart manager offers functionality like the import of data, geocoding of addresses and do anaylses like catchment areas and driving distance zones. To do all that analyses a lot of data is indispensable  like routing networks or information about the behaviour of potential costumers. All these data are offered by microm so that their costumers can buy the information they need. Doing so microm profits from their affiliation to Creditreform which collects costumer data from different resources among other things. As a subsidiary of Creditreform microm can use the information which Creditreform have been collected.

\subsection{Motivation and Research Question}
During the last years the innovation of computer systems raised up so that today a lot of business processes are done by computers to make them more effectiv concerning time and costs. This progress is considerable in the field of Geomarketing, too. More and more software is implemented helping sales managements and business analysts in their decisions. Although already some software packages which provide analyses for Geomarketing strategies are available, the amount of them is really small. Additionally the most of the algorithms containing restrictions concerning several parameters. Per example it may be important that created territories within an area segmentation process become well balanced. Additionally some other constraints may be considered at the same time. The most important condition during area segmentation processes are well balanced, compact and contiguous territories. But no software exists which considers all three parameters. Always just one constraint is kept in mind. That is why the aim of that master thesis is the implementation of an algorithm which considers the three mentioned conditions. Therefore existing algorithms will be considered in detail to determine whether it is possible to find approaches that are promising for the application to Geomarketing strategies. If approaches can be determined the question needs to be answered which algorithm yields the best results concerning balanced, contiguous and compact territories. Afterwards an application of the chosen algorithm will be done to three Geomarketing strategies: area segmentation and optimization, as well as Greenfield analyses and Whitespot analyses.

\subsection{Methods}
For finding a promising algorithm that can be applied to Geomarketing strategies at first previous work will be considered. Therefore algorithms of the past has been analysed to define approaches which will be implemented. As soon as algorithms are determined they will be implemented in JAVA. The results are displayed with QGIS. Afterwards a comparison of the algorithms will be possible using different defined parameters that concern the predefined conditions from the field of Geomarketing. The result of the comparison will show the most promising algorithm that will be applied to area segmentation and optimization, Greenfield analyses and Whitespot analyses. Therefore the algorithm will be enhanced so that the requirements of the analyses will be satisfied. Afterwards an evaluation of the results is done considering advantages and disadvantages as well as algorithms of related work. 

\subsection{Outline}
Within the last decades several institutes and companies implemented area segmentation algorithms for doing their analyses. Getting an insight into some related works at first three examples will be considered within the following chapter. Some common used termini related to Geomarketing will be introduced in section 3 ''Fundamentals of area segmentation''. This section also includes notions and use cases which explain processes of area segmentation in more detail. Afterwards the owned knowledge will be used to consider algorithms from past in more detail to make a selection of promising algorithms possible. Comparing different types of approaches supports the election of approaches. The comparison of approaches and the selection of promising algorithms will be done in section 4 ''Selecting approaches for implementation''. As soon as the selection is done the implementation of the algorithm will be done in section 5. Thereby parameters and data are considered first. Afterwards each algorithm will be explained in more detail, showing their procedure and presenting the results. Using these information a comparison of the implemented algorithms will be done in section 6. Consequently the most promising algorithm can be determined to apply it to Geomarketing strategies. The application of the algorithm to area segmentation and optimization, Greenfield analyses and Whitespot analyses will be shown within the following section. Based on the implementation an evaluation will be accomplished in section 8. Finally the work will be summarized to discuss the results and to state a perspective of further work.