\section{Evaluation}\label{evaluation}

After all implementations were done an evaluation of the used algorithm was accomplished. Therefore different investigation areas are used applying several parameters to make an analysis of the results possible. The results have been compared with results from other Geomarketing software. \\
First a comparison to a calculation in SIM Tool was done. Therefore microm provided some results doing a Greenfield analysis to zip-code8 areas of Hamburg. The area segmentation process was done using approximately 2500 basic areas. Thereby 200 new locations should be created. Afterwards the basic areas should be allocated to these locations. The calculation using SIM runs 1:30h. The same calculation was done using the Greenfield algorithm which was implemented within this thesis. For making the results comparable the same dataset was used. The aim of the implemented algorithm was to create coherent, compact and well balanced territories within an adequate running time. The results of the Greenfield analysis using the algorithm implemented for this thesis achieved the result within 5 minutes.. Contrary to the territories which were created by SIM, the created territories were all contiguous. Additionally the territories were more balanced than the ones of SIM. Nevertheless the balance can be optimized further. The result of the calculation using the algorithm implemented in this thesis is shown in Figure 47.

\begin{figure}[H]
	\centering
	\includegraphics[width=0.8\textwidth]{images/HH10.jpg}
	\caption[Result of Greenfield analysis doing to zip-code8 areas of Hamburg.]{Result of Greenfield analysis doing to zip-code8 areas of Hamburg. 10 locations and territories are created. The check of unity was not used in that calculation.}
\end{figure}


Additionally a comparison to another Geomarketing software called Map\&Market was done. Thereby whole Germany was used as investigation area, consequently approximately 8000 basic areas were assigned to the locations. The area segmentation was done using a Greenfield analyses creating 10 new locations. Doing this calculation a lot of problems occurred. The biggest problem are the islands in the North Sea and Baltic Sea. The islands are not connected to any other basic areas. Consequently they are not continuous to a territory in sense of the definition used in that thesis because no shared edges of two basic areas exists. That is why the check of contiguous yields to errors. To solve that problem a variable was integrated that shows whether the check of coherence should be done or not. In case of using islands within the amount of basic areas no totally contiguity can be reached thus the variable needs to indicate that no check will be done. Although the check of coherence was disabled during the calculation most of the basic areas are allocated in such a way that they are contiguous. The algorithm was improved thus the possibility creating incoherent territories become smaller. Thus the results form an expected output. The result is illustrated in Figure 48. 

\begin{figure}[H]
	\centering
	\includegraphics[width=0.9\textwidth]{images/dtl.jpg}
	\caption[Result of Greenfield analysis doing to zip-code5 areas of Germany.]{Result of Greenfield analysis doing to zip-code5 areas of Germany. 10 locations and territories are created. The check of unity was not used in that calculation. Parameters: compactness 1, balance 0, threshold 30.}
\end{figure}

Although a lot of improvements concerning the performance were done the calculation time applying the Greenfield algorithm to whole Germany is still too high. This is caused by two facts. One fact are the number of database accesses which are necessary during the calculation. Before the area segmentation starts the polygons including their properties and neighbouring relationships are stored into local variables. Using 8000 basic areas these preparatory works needs up to half an hour. After this the allocation starts which needs running time again. Especially the local search needs a lot of time, thus the calculation time is higher than the one of Map\&Market. Map\&Market needs 50 minutes, the algorithm which was implemented here needs 1:20h. Nevertheless the approach of the thesis owns some advantages compared to the one used in Map\&Market. Thus the compactness is considered and contiguity will be achieved if the data is applicable doing so. However some more improvements are still necessary. Some preparation within the database may yield to faster calculations as well. During the calculation of the two test cases some problems could be identified especially concerning performance. To make the algorithm much faster a lot of improvements were done. First of all the number of accesses to the database were reduced. Therefore a lot of queries were removed. Instead of this just one access to the database in the beginning is used now which stores all necessary data. This one is just one example of enhancement that were done. \\
Additionally to the two comparisons of use cases the algorithms were tested to different investigation areas using several parameters. Thereby some weaknesses could be determined. Especially the implementation of the local search yields to some problems. They are caused by rearranging same basic areas again and again so that no better balance will be achieved. That is why an abortion was implemented after a determined running time or number of rearrangements. This problems leads to two weaknesses. First it may be happen that the predefined balance threshold will be not reached. Consequently the territories are not well balanced. Second the running time is maybe still high although no better balance is achieved. That is why an advancement of that algorithm is still necessary in future work. However the implemented algorithms for area segmentation, Whitespot analyses and Greenfield analyses are good prototypes doing area segmentation. Compared to other applications they yield to convincing results within an adequate running time. Nevertheless some improvements are necessary concerning the balance and the performance. Additionally it is not satisfied that an optimal solution of area segmentation is found, but therefore the running time is still fast. 

