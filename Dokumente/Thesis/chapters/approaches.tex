\section{Selecting approaches for implementation}

\subsection{Overview about existing approaches}

\subsection{Selection of approaches}



A completely different allocation approach is to sequentially assign basic areas to territory
centers based on distance, i.e. a basic area will be allocated to closest territory center. This
minimal distance allocation yields disjoint, compact and often connected, however, usually
not well balanced territories as the balance criterion is completely neglected when deciding
about the allocation. The attractiveness of this method, denoted as AllocMinDist, primarily
lies in its simplicity and computational speed. See Kalcsics et al. [KMNG02].

\section{Implementation of area segmentation approaches}

\subsection{Segmentations considering homogenous distribution}

\paragraph{Considering just Criteria}

\paragraph{Considering average criteria value in proportion to number of locations}

\subsection{Segmentations considering distance}

\paragraph{Considering just Distance}

\subsection{Segmentations considering criteria and distance}

\paragraph{Criteria and Distance: from inside to outside: SmallestCritGetsNearest}

\paragraph{Criteria and Distance: from inside to outside: SmallestCritGetsTrueNearest}

\paragraph{Criteria and Distance: from outside to inside + inside to outside}

\paragraph{sum of criteria divided by number locations + Distance}

\paragraph{Distance + criteria}

