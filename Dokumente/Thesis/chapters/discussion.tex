\section{Discussion and Perspective}
\subsection{Summary}

Within that master thesis area segmentation processes were considered.Therefore in the beginning some relation works were described. Afterwards basic knowledge of the topic were explained additionally to criteria that are often used. Especially contiguity, balance and compactness were illustrated in more detail because these criteria are applied most o the time. Because of the application of the implemented algorithm to geomarketing analyses these three criteria will be considered here, too. For determining promising algorithm that will be implemented previous work and approaches were examined. Thereby three different types of models could be identified. The model types are called location-allocation approaches, set-partitioning approaches and heuristic algorithms. For each type advantages and disadvantages were shown in section \ref{comparison} \hyperref[comparison]{Comparison of model types} so that in conclusion the heuristic approaches could be defined as most usable algorithm for the application to geomarketing analyses that will be applied in that master thesis. Afterwards existing heuristics are surveyed. Additionally some new ones were developed by combining or improving existing approaches so that in the end 8 different algorithm were defined for implementation. Before the implemented algorithms were described at first some information about the investigation are and the data were given. Afterwards each algorithm was explained including its workflow and results. As soon as all algorithm were implemented a comparison of them was done in section \ref{comparisonapproaches} \hyperref[comparisonapproaches]{Comparison of implemented approaches}. Thereby three variables were defined that denote the criteria compactness, balance and contiguity. The comparison shows that the approach called AllocMinDistLocalSearch yields to the best results concerning the three parameters. Nevertheless some improvements were necessary to satisfying the requirements from the field of geomarketing during the calculation. Per example a check of contiguity needed to be implemented to get coherent territories. The improvements were integrated into the algorithm so that an application to geomarketing analyses was possible. The algorithm was used calculating area segmentations and optimizations, Greenfieldanalyses and Whitespotanalyses.All three types of calculations yields to good results. However some weaknesses were identifiable during the application so that additionally improvements are necessary within the future. These were discussed in section \ref{evaluation} \hyperref[evaluation]{Evaluation}. Additionally in that section the results were compared to ones provided by other geomarketing software. That confrontation shows some advantages of the implemented algorithm per example the check of contiguity or the performance of some test cases. It shows that the goal of developing an usable algorithm which yields some advantages compared to other algorithms was reached. At the same time it is recognizable that some enhancements are necessary to make the calculation more stable and dependable.

\subsection{Perspective}
The comparison of the implemented algorithm to existing geomarketing tools showed that the results are really promising. The calculations runs in the most of the cases faster and the results are looking more convincing. Especially the implementation of a used threshold value and a weighting value of balance and compactness presents a great improvement compared to other tools. However, some additional enhancement will be necessary within the future. The greatest problem of the implemented algorithm is the realization of the local search. It is possible that always the same basic areas will be rearranged without getting better balanced value. Consequently it is needed to find an approach which do not show that problem and yields to balanced territories always. It may be recommendable to compare the results to one other approach for example the one implemented by \citeauthor{kalcsics} \cite{kalcsics}. In his work an algorithm was developed using consistent tree decomposition which is based on an location-allocation approach. The disadvantage of the implemented heuristic approach of that thesis is that just one solution will be determined. Within the location-allocation approach the best resolution will be calculated. Consequently a comparison of the results of both algorithms would be advisable to evaluate the quality of the results of the heuristic approach. It is recommend to analyse which approach yields to the best result considering the ratio of quality and running time. It was already mentioned in section \ref{comparison} \hyperref[comparison]{Comparison of model types} that location-allocation algorithms need a high running time to get a result of the area segmentation. It may be important to compare the running time on the basis of a specific example to proof that conclusion. \\
Furthermore the application of the algorithm to a dataset which considers whole Germany shows that problems exist if island are used as basic areas. In such cases it is necessary at the moment to turn off the check of contiguity. In future it will be a good idea finding an approach which considers coherence also if island are used. Especially the check of contiguity is one of the great advantages of that algorithm to other tools, consequently it should be usable within mostly all investigations. But the check of coherent territories needs a lot of performance so that the running time raises up if huge territories are created. That is why there also an improvement is needed to get a satisfying result within an adequate running time.\\
The area segmentation is based on an allocation of basic areas by distance. Surveys show that using just linear distances may yield to results which no geomarketing analyst would create in reality because geographical barriers as mountains or rivers would be considered during the creation by hand. That is why in future besides linear distance additionally driving distance and driving time will be considered as well. \\
As soon as all improvements are implemented the prototype of the developed algorithm of that thesis will be a powerful and dependable tool doing geomarketing analyses effectively. The developed prototype is a good base to simplifies analyses from the field of geomarketing for analysts to use it within a software tool.
