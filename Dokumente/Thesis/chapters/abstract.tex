\section*{Abstract}

Already since several years companies are doing geomarketing analyses to support their decisions raising their profit and minimizing risks during their product and location planning. Nowadays geomarketing strategies getting more and more important because more companies realize that they are a helpful tool. To make them more effective concerning time and cost the analyses are done more and more using computers. But at the moment just a small amount of tools are available which offer functionality of geomarketing strategies to their costumers. Often the available software considers not all requirements that are defined by geomarketing experts. For doing area segmentation the most common conditions are that the created territories need to be balanced, compact and contiguous. Area segmentation processes are clustering small geographic areas to bigger units which are called territories. Motivated by the small amount of available software and that the most important conditions of area segmentation are not kept in mind the goal of that master thesis is the development of an algorithm doing area segmentation while considering balanced, compact and coherent territories. Therefore previous work will be analysed for determining promising algorithm. These algorithm will be implemented. Afterwards the results are compared determining the algorithm with the most promising results. That algorithm will be applied to three geomarketing strategies: area segmentation and optimization, Greenfield- and Whitespotanalyses. At the end the results will be evaluated and compared to previous work showing advantages and possible disadvantages if exists. 