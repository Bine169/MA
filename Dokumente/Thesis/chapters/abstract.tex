\section*{Abstract}

Already since several years companies are doing Geomarketing analyses to support their decisions raising their profit and minimizing risks during their product and location planning. Nowadays Geomarketing strategies getting more and more important because more companies realize that they are a helpful tool. To make them more effective concerning time and cost the analyses are done more and more by using computers. But at the moment just a small amount of tools are available which offer functionality of Geomarketing strategies to their costumers. Additionally often the available software considers not all requirements that are defined by Geomarketing experts during the application of area segmentation. Area segmentation processes are clustering small geographic areas to bigger units which are called territories. Doing so the most common conditions are that the created territories need to be balanced, compact and contiguous.  Motivated by the small amount of available software and that the most important conditions of area segmentation are not kept in mind the goal of this thesis is the development of an algorithm supporting area segmentation while considering balanced, compact and coherent territories. Therefore previous work will be analysed for determining promising algorithms. These algorithms will be implemented in order to compare them and determining the most promising approach. The algorithm with best results will be applied to three Geomarketing strategies: area segmentation and optimization, Greenfield analyses and Whitespot analyses. Finally the results will be evaluated and compared to previous work showing advantages and possible disadvantages if exist. 